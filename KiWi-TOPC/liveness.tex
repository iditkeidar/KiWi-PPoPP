

\subsection{Liveness}
\label{sec:live}

We now prove \kiwi's liveness properties. 
We show that gets and scans are \emph{wait-free}, namely, in any execution, every operation completes within a finite number of steps by its invoking thread. 
 This is proven  by showing that the number of iterations in the loops in these procedures is finite. 
 Put operations satisfy a weaker liveness property --  \emph{lock-freedom}, namely, in every execution, \emph{some} put operation completes.
To prove this, we show that although a put operation can execute an infinite number of rebalances, this can only occur because 
some other operation (and in fact many operations) successfully complete a put.

We begin by showing that the \codeF{locate} function is wait-free. 
\begin{proposition}  
The  \code{locate} function is wait-free.
\label{prop:locate}
\end{proposition}
\begin{proof}
The function consists of two loops. The first loop 
executes a finite number of iterations because \code{index.lookup(key)} always returns a chunk $C_0$ with $C_0$.\code{minKey} $\le$ key,
and so the searched keys are monotonically decreasing.  Since the number of keys is finite and the head of the list is never frozen, the loop
returns an unfrozen chunk after a finite number of iterations. The ensuing traversal is also finite because the linked list is finite. 
\end{proof}

\begin{lemma}
The  \code{get} and \code{scan} functions are wait-free.
\end{lemma}
\begin{proof}
By Proposition~\ref{prop:locate}, the \code{locate} function is wait-free. 
To complete the proof, observe that the loops performed in \code{helpPendingPuts} and \code{findLatest}
 iterate (or search) over finites sets -- $C.k$ and $C.ppa$, and the scan loop (line~\ref{l:scan-loop}) traverses
 chunks in the (finite) linked list. 
\end{proof}

\begin{lemma}
The  \code{put} function is lock-free.
\end{lemma}
\begin{proof}
We first show that as long as no \code{rebalance} occurs, \code{put} completes within a finite number of steps. 
Beyond \code{rebalance} calls and recursive calls to \code{put} when it fails, \code{put} executes 
two loops --  one in the \code{locate} function and one 
%only one loop, 
in phase 3 (lines~\ref{l:put-ll}--\ref{l:put-ll-end}).  
The former is wait-free by  Proposition~\ref{prop:locate}. 
Next, consider the loop in phase 3 of the \code{put}. 
Observe that the \codeF{find} function is wait-free because it searches a finite array ($C$.k).
Moreover, the CAS in line~\ref{l:put-insert1} fails at most a finite number of times (again, due to the finality of  $C$.k),
and because we break when it succeeds, it is attempted a finite number of times. 
Additionally, every time the CAS in line~\ref{l:put-insert2} is attempted (and either succeeds or fails),   \codeF{c.valPtr} increases,
implying that after a finite number of attempts, {\codeF{c.valPtr} $ \geq$ \codeF{j}}, and the loops completes. 

Next, consider \code{rebalance} triggered by the \code{put}. This can occur in \code{checkRebalance} (called in line~\ref{l:put-rebalance0}), 
or when either $C.k$ or $C.v$ is full (line~\ref{l:put-full}), or when the chunk is already frozen (line~\ref{l:put-restart}). 
To prove lock-freedom, 
we will show that every time \code{rebalance} returns false (leading to a recursive call to \code{put}) or iterates  
an additional time in an unbounded loop, or recursively calls \code{rebalance}, this is because some new  \code{put} returned successfully. 
Note in particular, that if a  \code{rebalance} operation succeeds to replace the engaged sequence (and returns true), then it 
completes the \code{put} that invoked it, and so we do not need to consider this case.

We first argue that with the exception of rebalance stage 5 (replace), all other  \code{rebalance} stages  always complete within a finite number of steps. 
Because the chunks list is finite, the two loops in stage 1 (engage) are finite. 
Similarly, the list of engaged chunks and the number of PPA entries in each of them are finite, and so is the nested loop in stage 2 (freeze). 
In stage 3 (pick minimal version), the loop over the PSA is finite, and it constructs a finite toHelp set, and so the loop over toHelp is finite as well.
Likewise, the loops in stage 4 iterate over finite sets, namely, engaged chunks and keys in the data structures therein.  
In stage 6, the conditional insertion to the index fails only if another thread succeeds to insert a chunk with the same \code{minKey}. 
Because the insertion is retried only as long as the chunk is not frozen, interfering insertions to the index must be due older rebalances that began before 
the current rebalance, of which there is  a finite number.
Clearly, stage 7, which performs a single CAS, is also wait-free. 
Hence, it remains to consider stage 5.   

The first loop in stage 5 terminates once the CAS -- either of the thread executing the loop or of another thread -- succeeds to mark \code{last.next} as immutable. 
The CAS fails if either \code{last.next} changes or another thread marks it.  
But  \code{last.next} only changes if some \code{rebalance} succceds to replace an engaged chunk succeeding it, and this  \code{rebalance} 
returns true upon completion, thus completing a \code{put} when it returns. 

The second loop returns whenever either the current thread or another thread successfully replaces the \code{next} pointer at C's predecessor to 
a node whose parent is $C$ ($C_f$ in case it is the current thread). 
They fail to do so only in case the predecessor's  \code{next} pointer is marked, which in turn only occurs if the predecessor itself is engaged in 
another rebalance. In this case, the current thread makes a recursive call to  \code{rebalance} in order to help the predecessor (line~\ref{l:nested-rebalance}), 
and then executes another iteration of the loop. 
Note that since the linked list is finite, the depth of the recursion is finite, so eventually the inner-most nested recursive call succesfully returns. 
Once a nested call in line~\ref{l:nested-rebalance} returns after having replaced \code{pred} with a new chunk \code{pred'}, 
either some thread will succeed to CAS the new predecessor's next pointer causing 
the calling thread to exit the loop in the next iteration, or  the new chunk \code{pred'}  will undergo rebalance.
However, for the latter to occur, it must be the case that at least one new \code{put}  has completed in  \code{pred'} following its rebalance.
This is because when rebalance completes, \code{pred'.k} is sorted and  \code{pred'.k} and \code{pred'.v} are at most half full, and 
because they hold more entries than twice the number of threads,   it is impossible for either of them to fill up due to pending put operations. 
Thus, \code{rebalance} can only be required after a successful   \code{put} in   \code{pred'}.

\remove{

We prove by contradiction.
Assume that \emph{put} is not lock-free.
Hence there exists an infinite execution $\pi = C_0,s_1,C_1,\ldots$ such that
after configuration $C_{k0}$ ($k0 \geq 0$) no operation completes and no new operation is invoked.
%no operation completes after configuration $C_{k0}$.
%
We have already shown that \emph{get} and \emph{scan} are wait-free,
 therefore there exists configuration $C_{k1}$ in $\pi$ ($k0 \leq k1$) such that after $C_{k1}$ all the running threads execute  \emph{put} operations.


We write \emph{$t:X$.allocate} to denote an invocation of \emph{allocate} on chunk $X$ by thread $t$
(\emph{allocate} is invoked by function \emph{put} at line $6$).
We say that invocation \emph{t:$X$.allocate} is \emph{successful}
if it returns a reference to a valid cell (i.e., it returns a non-null value).

Consider an execution of the loop in function \emph{put} by thread $t$.
If \emph{$t:X$.allocate} is not successful at iteration $i+1$ ($i \geq 1$) of $t$ ,
then at iteration $i$ the invocation of \emph{$t:Y$.allocate} is also not successful.
Hence thread $t$ invokes \emph{Rebalance($Y$)} at the end of iteration $i$.
Therefore $Y \neq X$ and chunk $X$ has been added to kiwi (at some point) during iterations $i$ and $i+1$ of $t$
(when a chunk is added to kiwi, this chunk is not full).
Therefore another thread $t' \neq t$ successfully invoked \emph{$t':X$.allocate} (at some point) during iterations $i$ and $i+1$ of $t$
(otherwise \emph{$t:X$.allocate} should be successful).
%
Since a thread do not start new iteration of this loop after a successful invocation of \emph{allocate} (see lines $6$--$9$ in \emph{put}),
 we know that there exists configuration $C_{k2}$ in $\pi$ ($k1 \leq k2$) such that:
after $C_{k2}$ the \emph{put} operations do not start new iterations of this loop.

Since \emph{allocate} is wait-free, there exists configuration $C_{k3}$ in $\pi$ ($k2 \leq k3$) such that
no thread executes \emph{allocate} after $C_{k3}$.
Hence, no chunk becomes full in $\pi$ after configuration $C_{k3}$.


%Consider the loop in function \emph{put}.
%This loop may have iteration $i+1$ only if \emph{chunk.allocate} is not successful at iteration $i$ ---
%this can only happen if (at some point) after the beginning the $i$-th iteration a concurrent \emph{put} operation has successfully invoked \emph{chunk.allocate} on the same chunk.
%Hence there exists configuration $C_{k2}$ in $\pi$ ($k1 \leq k2$) such that: after $C_{k2}$ the \emph{put} operations do not start new iterations of this loop.

In the following paragraphs we focus on the functions \emph{Kiwi::Rebalance} and \emph{Kiwi::Replace}.
Notice that \emph{Kiwi::Rebalance} calls to \emph{Kiwi::Replace}; and that \emph{Kiwi::Replace} recursively calls to \emph{Kiwi::Rebalance} in line $20$.
An invocation of \emph{Kiwi::Rebalance} on chunk $X$ removes $X$ from kiwi:
hence for each thread $t$ and chunk $X$, \emph{Rebalance($X$)} may be invoked at most once by $t$.

Let $N_{puts}$ be the number of \emph{put} operations after configuration $C_{k3}$.
Let $N_E$ be the maximal number of new chunks created by an invocation of \emph{balance()} (in our implementation $N_E=4$).
Since after $C_{k3}$ no chunk may become full, at most $N_{puts} \times N_E$ new chunks may be created after $C_{k3}$.
Therefore, after configuration $C_{k3}$ the threads invoke function  \emph{Kiwi::Rebalance} a finite number of times.
Hence there exists a configuration $C_{k4}$ in $\pi$ ($k3 \leq k4$) such that \emph{Kiwi::Rebalance} is not invoked after $C_{k4}$.



%Let $N_{puts}$ be the number of \emph{put} operations after configuration $C_{k2}$.
%After $C_{k2}$, at most $N_{puts}$ chunks may become full.
%If $N_E$ is the maximal number of new chunks created by an invocation of \emph{balance()},
%then after configuration $C_{k2}$ at most $N_{puts} \times N_E$ new chunks may be created.
%Therefore, after configuration $C_{k2}$ the threads invoke function  \emph{Kiwi::Rebalance} a finite number of times.
%Hence there exists a configuration $C_{k3}$ in $\pi$ ($k2 \leq k3$) such that \emph{Kiwi::Rebalance} is not invoked after $C_{k3}$.

%Consider the loop in \emph{Kiwi::Replace}.

Since the other functions invoked by \emph{put} are lock-free\footnote{
The other functions invoked by \emph{put}, \emph{Kiwi::Rebalance} and \emph{Kiwi::Replace} are lock-free:
we assume that \emph{index.replace} is lock-free, the other ones are trivially wait-free.},
there exists a configuration $C_{k5}$ in $\pi$ ($k4 \leq k5$) such that all the running threads
after $C_{k5}$ execute the loop in \emph{Kiwi::Replace} (and this loop is never terminated).
%
Hence, the CAS at line $22$ always fails after configuration $C_{k5}$ (i.e., no CAS updates shared memory after $C_{k5}$).
This is a contradiction, because the  CAS at line $22$ may fail only if shared memory has been updated since the beginning of the iteration.



%
%
%
%Consider the loop in \emph{Kiwi::Replace}.
%Since the other functions are lock-free, there exists a configuration $C_{k4}$ in $\pi$ ($k3 \leq k4$) such that all the threads which are running after $C_{k4}$ do not exit from this loop.
%Hence, the CAS at line $22$ always fails after configuration $C_{k4}$ (i.e., no CAS updates shared memory after $C_{k4}$).
%This is a contradiction, because the  CAS at line $22$ may fail only if shared memory has been updated since the beginning of the iteration.
}
\end{proof}
