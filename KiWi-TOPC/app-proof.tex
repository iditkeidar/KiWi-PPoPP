\section{Proof Appendix}
\label{app:proof}

\subsection{Safety}
\label{app:proof:safety}

%\par{Lemma~\ref{proof:put} Proof.}
\begin{proof} \textbf{of Lemma~\ref{proof:rebalance}.}
\end{proof}

\begin{proof} \textbf{of Lemma~\ref{proof:put}.}
We consider an execution interval $\pi$ which spans the execution intervals of all operations in $OP_k^c$.
Denote by $\sigma_1, \sigma_2, \ldots $ the finite sequence of steps of these operations writing versions $v_1, v_2, \ldots$ into entries in the \code{ppa} by their order in $\pi$, where $\sigma_i$ is a step of operation $op_i$; the locations each operation allocated for its value are $j_1, j_2, \ldots$, respectively.

The proof is by induction on $i$. For the base case, we consider $op_1$. It is the first to publish its version in the \code{ppa}. Lemma~\ref{lemma:rebalance:infant} implies that all the cells within the chunk range that were inserted into an earlier chunk were added to the chunk's cell linked list by a rebalance operation that completed before $op_1$ started. Therefore, these cells have smaller location-based versions and $op_1$ is linearized in the last read step of the global version counter that returns $v_1$ before $\sigma_1$. Clearly this step is after the put operation is published, and specifically after $j_1$ is allocated, and the lemma holds.

For the induction step, assume the lemma holds for operations $op_1, \ldots op_{i-1}$. We prove the lemma for operation $op_i$ by case analysis.
If $op_i$'s location-based version $\langle v_i, j_i\rangle$ is maximal in $C$ (with respect to all linked cells and published entries with the same key) then \lp{op_i} is the last read retrieving $v_i$ from the global counter. This step is done after the put is published in the \code{ppa} (which is after $j_i$ is allocated) and before $\sigma_i$, and Conditions~\ref{proof:put:lp1}-\ref{proof:put:lp3} of the lemma hold. In addition, by the induction hypothesis, linearization points of $op_1, \ldots op_{i-1}$ preserve their location-based version order. They are all linearized in read steps of the global version counter returning their versions, specifically not later than \lp{op_i}---the latest read step returning the maximal version, hence Condition~\ref{proof:put:lp4} holds as well.

Otherwise, another operation $op_l$ published $\langle v_l, j_l\rangle$ in $\sigma_l$ before $\sigma_i$, s.t. $\langle v_l, j_l\rangle > \langle v_i, j_i\rangle$. By definition, $op_i$ is linearized exactly at the point (\lp{op_l}) which preserves the location-based version order of the operations, and Condition~\ref{proof:put:lp4} holds. By Condition~\ref{proof:put:lp3} of the induction hypothesis, \lp{op_l} is after an operation in $OP_k^c$ with location-based version equal or greater than $\langle v_l, j_l\rangle$ is published in $c$'s \code{ppa}. Since $\langle v_l, j_l\rangle > \langle v_i, j_i\rangle$, Condition~\ref{proof:put:lp3} also holds.

It is left to discuss Conditions~\ref{proof:put:lp1} and~\ref{proof:put:lp2}. Consider first the case where $v_l>v_i$. By Condition~\ref{proof:put:lp2} of the induction hypothesis \lp{op_l} is in a read step of the global version counter that occurred after it is set to $v_l$. $op_i$ eventually obtains the version $v_i$ which is smaller than $v_l$. This implies $op_i$ published the operation in the \code{ppa} before the version counter is set to $v_l$, and  \lp{op_i} satisfies Conditions~\ref{proof:put:lp1} and~\ref{proof:put:lp2}. If $v_l =v_i$ and $j_l>j_i$ then $op_l$ allocated $j_l$ after $op_i$ allocated $j_i$. By Condition~\ref{proof:put:lp1} of the induction hypothesis, \lp{op_l} is after $op_l$ allocated $j_l$  and before $\sigma_l$, and  \lp{op_i} satisfies Conditions~\ref{proof:put:lp1} and~\ref{proof:put:lp2}.
\end{proof}

%\par{Lemma~\ref{proof:get} Proof.}
\begin{proof}\textbf{of Lemma~\ref{proof:get}.}
It can be inferred from \code{findLatest} that $op$ returns the value with the maximal location-based version observed by $op$ in the \code{ppa} and in the cell linked list. In addition, it can be inferred from the code that put operations update values in-place in the cell linked list only if its location-based version is higher than the location-based version of the cell in the list.

First, assume $op$ did not find the key in $c$. By the observations above, all operations in $OP_k^c$ are published in the \code{ppa} after $\tau$. Otherwise, $op$ should have observed them either in the \code{ppa} or in the linked list. By Condition~\ref{proof:put:lp3} of Lemma~\ref{proof:put}, all operations in $OP_k^c$ are linearized after $\tau$, and Condition~\ref{proof:get:lp1} holds.

Next, assume $op$ returns the value written by operation $op_l$;
the location-based version of $op_l$ is $\langle v_l, j_l\rangle$. 

If $c$ is not accessible from the chunks list in the configuration preceding $\tau$, then \lp{op} is in the last step in which $c$ is accessible from the chunks list. By Claim~\ref{proof:rebalance:freezing} of Lemma~\ref{proof:rebalance} no put operation can publish its location-based version in $c$'s \code{ppa} after $\tau$, and by Condition~\ref{proof:put:lp1} of Lemma~\ref{proof:put}  \lp{op} is after  \lp{op_l}. Otherwise, it is clear by definition that \lp{op} is after  \lp{op_l}.

Consider an operation $op_m \in OP\setminus\{op_l\}$ with location-based version $\langle v_m, j_m\rangle$.
By Condition~\ref{proof:put:lp4} of Lemma~\ref{proof:put}, if $\langle v_m, j_m\rangle < \langle v_l, j_l\rangle$ then \lp{op_m} is before \lp{op_l}. It is left to show that if $\langle v_m, j_m\rangle > \langle v_l, j_l\rangle$ then \lp{op_m} is after $\tau$. By the observations above, all operations $op_m \in OP_k^c\setminus\{op_l\}$ with location-based version $\langle v_m, j_m\rangle > \langle v_l, j_l\rangle$ are published in the \code{ppa} after $\tau$, and hence are not observed by $op$. Condition~\ref{proof:put:lp3} of Lemma~\ref{proof:put} implies that these operations are linearized after at least one of them is published in the \code{ppa}, hence Condition~\ref{proof:get:lp2} also holds.
\end{proof}

%\par{Lemma~\ref{proof:scan} Proof.}
\begin{proof}\textbf{of Lemma~\ref{proof:scan}.}
It can be inferred from \code{findLatest} that for each key $k$ in the range of the scan, $op$ returns the maximal location-based version of $k$ that does not exceed the scan read point observed by $op$ in the \code{ppa} and in the cell linked list. In addition, it can be inferred from the code that put operations update values in-place in the cell linked list only if its location-based version is higher than the location-based version of the cell in the list.

Consider a put operation $op_m$ that writes to a key in the range of the scan but is not observed by $op$. If $op_m$ acquires version that is less than $v$ then it acquired a version before the scan increased the global version counter. Since $op$ did not observe $op_m$ in the \code{ppa}, $op_m$ completed before $op$ read the entry in the \code{ppa}, and since $op$ did not observe $op_m$ in the linked list, then the location-based version of the cell with key $k$ already in the linked list is higher than the location-based version of $op_m$.
By Condition~\ref{proof:put:lp4} of Lemma~\ref{proof:put}, $op_m$ is linearized before the put operation writing the value returned by the scan.

Finally, by Condition~\ref{proof:put:lp2} of Lemma~\ref{proof:put} all put operations that are not observed by $op$ and acquire version that is greater than $v$ are linearized after \lp{op}.
\end{proof}


\subsection{Progress}
\label{app:proof:progress}

\newcommand{\abs}[1]{\mid{#1}\mid}

We now show that \emph{get} and \emph{scan} operations are wait-free, and \emph{put} operations are lock-free.

We say that a sequence $\pi_s$ is a \emph{sub-execution}, if $\pi_s$ is a suffix (or a prefix) of an execution.
We say that thread $t$ is \emph{running} in a sub-execution $\pi_s$, if at least one step in $\pi_s$ is executed by thread $t$.
%
Given two keys $K_1$ and $K_2$, we write $\abs{K_2 - K_1}$ to denote the number of possible keys between $K_1$ and $K_2$ (notice that $\abs{K_2 - K_1}$ is always a finite number because each key is stored in a bounded number of bytes).
%
Given a chunk $X$, we write $min(X)$ to denote the minimal key in $X$ (as mentioned before, $min(X)$ is never changed during the lifetime of $X$).



\begin{lemma}
\label{proof:orderLemma}
Let $\pi = C_0,s_1,C_1, \ldots,s_i,C_i,\ldots$ be an execution.
Let $X$ and $Y$ be two chunks in configuration $C_i$ such that $Y=X.next$.
For any configuration $C_j$ such that $j \geq i$ we have  $min(X) < min(Y)$ in $C_j$.
\end{lemma}
\begin{proof}
The algorithm writes the address of $Y$ in $X.next$ only if $min(X) < min(Y)$
(this happens either at line $22$ of the function \emph{Kiwi::Replace}, or within the function \emph{balance()}).
Since $min(X)$ and $min(Y)$ are never changed, $min(X) < min(Y)$ in  any $C_j$ such that $j \geq i$.

\end{proof}

\begin{lemma}
\label{proof:aaaaaaaa}
The function \emph{get} is wait-free.
\end{lemma}
\begin{proof}
We prove that \emph{get} is wait-free by showing that the functions \emph{Kiwi::FindChunk} and \emph{Chunk::Find} are wait-free.

The function \emph{Kiwi::FindChunk} has a single loop: we write ${next}_i$ to denote the value of variable \emph{next} at the beginning of the $i$-th iteration of the loop.
%
Because of Lemma~\ref{proof:orderLemma}, $min({next}_i) < min({next}_{i+1})$.
Hence the loop terminates after at most $\abs{key - min({next}_1)}$ iterations.
%
Since all the functions invoked by \emph{Kiwi::FindChunk} are wait-free, each invocation of \emph{Kiwi::FindChunk} by thread $t$ completes within a finite number of steps by $t$.

The function \emph{Chunk::Find} invokes the functions \emph{Chunk::FindInPendingList} and \emph{Chunk::FindInCellList}.
The loop in \emph{Chunk::FindInPendingList} completes after a constant number of iterations.
The function \emph{Chunk::FindInCellList} goes over the chunk's link-list exactly once (this link-list has at most $M$ cells).
Hence, each invocation of \emph{Chunk::Find} by thread $t$ completes within a finite number of steps by $t$.

\end{proof}



\begin{lemma}
\label{proof:aaaaaaaa}
The function \emph{scan} is wait-free.
\end{lemma}

\begin{proof}
All functions invoked by \emph{scan} are wait-free.
It is sufficient to show that the loops in  \emph{scan} have
a finite number of iterations.

The function \emph{scan} has two nested loops:
we write $L_e$ to denote the external loop (begins at line $10$), and $L_i$ to denote the internal loop (begins at line $12$).
Let ${key}_i$ be the value of variable \emph{key} at the end of the $i$-th iteration of $L_e$.
Each ${key}_i$ is equal to the minimal key of the chunk which is handled by iteration $i+1$
(this chunk is pointed by the variable \emph{chunk}).
Because of Lemma~\ref{proof:orderLemma}, if $L_e$ has (at least) $i+1$ iterations then ${key}_i < {key}_{i+1}$.
%Since the chunks are sorted according to their minimal keys, ${key}_i < {key}_{i+1}$ for any $i$.
Hence, $L_e$ has at most  $\abs{maxKey - minKey}$ iterations.

The loop $L_i$ goes over the link-list of a chunk (this link-list has at most $M$ elements), hence $L_i$ has at most $M$ iterations.

Therefore, each invocation of \emph{scan} by thread $t$ completes within a finite number of steps by $t$.

\end{proof}





\begin{lemma}
\label{proof:aaaaaaaa}
The function \emph{put} is lock-free.
\end{lemma}
\begin{proof}
We prove by contradiction.
Assume that \emph{put} is not lock-free.
Hence there exists an infinite execution $\pi = C_0,s_1,C_1,\ldots$ such that
after configuration $C_{k0}$ ($k0 \geq 0$) no operation completes and no new operation is invoked.
%no operation completes after configuration $C_{k0}$.
%
We have already shown that \emph{get} and \emph{scan} are wait-free,
 therefore there exists configuration $C_{k1}$ in $\pi$ ($k0 \leq k1$) such that after $C_{k1}$ all the running threads execute  \emph{put} operations.


We write \emph{$t:X$.allocate} to denote an invocation of \emph{allocate} on chunk $X$ by thread $t$
(\emph{allocate} is invoked by function \emph{put} at line $6$).
We say that invocation \emph{t:$X$.allocate} is \emph{successful}
if it returns a reference to a valid cell (i.e., it returns a non-null value).

Consider an execution of the loop in function \emph{put} by thread $t$.
If \emph{$t:X$.allocate} is not successful at iteration $i+1$ ($i \geq 1$) of $t$ ,
then at iteration $i$ the invocation of \emph{$t:Y$.allocate} is also not successful.
Hence thread $t$ invokes \emph{Rebalance($Y$)} at the end of iteration $i$.
Therefore $Y \neq X$ and chunk $X$ has been added to kiwi (at some point) during iterations $i$ and $i+1$ of $t$
(when a chunk is added to kiwi, this chunk is not full).
Therefore another thread $t' \neq t$ successfully invoked \emph{$t':X$.allocate} (at some point) during iterations $i$ and $i+1$ of $t$
(otherwise \emph{$t:X$.allocate} should be successful).
%
Since a thread do not start new iteration of this loop after a successful invocation of \emph{allocate} (see lines $6$--$9$ in \emph{put}),
 we know that there exists configuration $C_{k2}$ in $\pi$ ($k1 \leq k2$) such that:
after $C_{k2}$ the \emph{put} operations do not start new iterations of this loop.

Since \emph{allocate} is wait-free, there exists configuration $C_{k3}$ in $\pi$ ($k2 \leq k3$) such that
no thread executes \emph{allocate} after $C_{k3}$.
Hence, no chunk becomes full in $\pi$ after configuration $C_{k3}$.


%Consider the loop in function \emph{put}.
%This loop may have iteration $i+1$ only if \emph{chunk.allocate} is not successful at iteration $i$ ---
%this can only happen if (at some point) after the beginning the $i$-th iteration a concurrent \emph{put} operation has successfully invoked \emph{chunk.allocate} on the same chunk.
%Hence there exists configuration $C_{k2}$ in $\pi$ ($k1 \leq k2$) such that: after $C_{k2}$ the \emph{put} operations do not start new iterations of this loop.

In the following paragraphs we focus on the functions \emph{Kiwi::Rebalance} and \emph{Kiwi::Replace}.
Notice that \emph{Kiwi::Rebalance} calls to \emph{Kiwi::Replace}; and that \emph{Kiwi::Replace} recursively calls to \emph{Kiwi::Rebalance} in line $20$.
An invocation of \emph{Kiwi::Rebalance} on chunk $X$ removes $X$ from kiwi:
hence for each thread $t$ and chunk $X$, \emph{Rebalance($X$)} may be invoked at most once by $t$.

Let $N_{puts}$ be the number of \emph{put} operations after configuration $C_{k3}$.
Let $N_E$ be the maximal number of new chunks created by an invocation of \emph{balance()} (in our implementation $N_E=4$).
Since after $C_{k3}$ no chunk may become full, at most $N_{puts} \times N_E$ new chunks may be created after $C_{k3}$.
Therefore, after configuration $C_{k3}$ the threads invoke function  \emph{Kiwi::Rebalance} a finite number of times.
Hence there exists a configuration $C_{k4}$ in $\pi$ ($k3 \leq k4$) such that \emph{Kiwi::Rebalance} is not invoked after $C_{k4}$.



%Let $N_{puts}$ be the number of \emph{put} operations after configuration $C_{k2}$.
%After $C_{k2}$, at most $N_{puts}$ chunks may become full.
%If $N_E$ is the maximal number of new chunks created by an invocation of \emph{balance()},
%then after configuration $C_{k2}$ at most $N_{puts} \times N_E$ new chunks may be created.
%Therefore, after configuration $C_{k2}$ the threads invoke function  \emph{Kiwi::Rebalance} a finite number of times.
%Hence there exists a configuration $C_{k3}$ in $\pi$ ($k2 \leq k3$) such that \emph{Kiwi::Rebalance} is not invoked after $C_{k3}$.

%Consider the loop in \emph{Kiwi::Replace}.

Since the other functions invoked by \emph{put} are lock-free\footnote{
The other functions invoked by \emph{put}, \emph{Kiwi::Rebalance} and \emph{Kiwi::Replace} are lock-free:
we assume that \emph{index.replace} is lock-free, the other ones are trivially wait-free.},
there exists a configuration $C_{k5}$ in $\pi$ ($k4 \leq k5$) such that all the running threads
after $C_{k5}$ execute the loop in \emph{Kiwi::Replace} (and this loop is never terminated).
%
Hence, the CAS at line $22$ always fails after configuration $C_{k5}$ (i.e., no CAS updates shared memory after $C_{k5}$).
This is a contradiction, because the  CAS at line $22$ may fail only if shared memory has been updated since the beginning of the iteration.



%
%
%
%Consider the loop in \emph{Kiwi::Replace}.
%Since the other functions are lock-free, there exists a configuration $C_{k4}$ in $\pi$ ($k3 \leq k4$) such that all the threads which are running after $C_{k4}$ do not exit from this loop.
%Hence, the CAS at line $22$ always fails after configuration $C_{k4}$ (i.e., no CAS updates shared memory after $C_{k4}$).
%This is a contradiction, because the  CAS at line $22$ may fail only if shared memory has been updated since the beginning of the iteration.

\end{proof}



