\documentclass[acmsmall, review]{acmart}
\usepackage{url}                  % format URLs
%\usepackage[colorlinks=true,allcolors=blue,breaklinks,draft=false]{hyperref}   % hyperlinks, including DOIs and URLs in bibliography
\usepackage{amsmath}
\usepackage{xspace}
\usepackage{algorithm}
\usepackage[noend]{algpseudocode}
\usepackage{subcaption}
\usepackage{xcolor}
%\usepackage[pdftex]{graphicx}
\usepackage{tikz}
\usepackage{arydshln}
%\usepackage{acmcopyright}


\usepackage{pifont}
\usepackage{amssymb}

%\usepackage{titlesec}
\usepackage{multicol}

\usepackage{pgfplots}



\definecolor{blues1}{RGB}{198, 219, 239}
\definecolor{blues2}{RGB}{158, 202, 225}
\definecolor{blues3}{RGB}{107, 174, 214}
\definecolor{blues4}{RGB}{49, 130, 189}
\definecolor{blues5}{RGB}{8, 81, 156}
\definecolor{antiquefuchsia}{rgb}{0.57, 0.36, 0.51}
\definecolor{asparagus}{rgb}{0.53, 0.66, 0.42}
\definecolor{darkspringgreen}{rgb}{0.09, 0.45, 0.27}
\definecolor{darkslategray}{rgb}{0.18, 0.31, 0.31}
\definecolor{coralred}{rgb}{1.0, 0.25, 0.25}

\pgfplotsset{mystyle/.style={%
        width=0.35\textwidth,
        xmin=1,xmax=32,
        enlargelimits=true,
	ymajorgrids=true,
        grid=major,
        grid style={dashed, gray!30},
        ylabel style={align=center, font=\small},
        xlabel style={align=center, font=\small},
        xtick={2,4,8,16,32}, 
         x tick label style={font=\tiny},
         y tick label style={font=\tiny},
        scaled y ticks=base 10:-6,
        ytick scale label code/.code={},
        title style={font=\small},
      	yticklabel={\pgfmathprintnumber{\tick}}
}}

\pgfplotsset{rangestyle/.style={%
        width=0.35\textwidth,
        enlargelimits=true,
	ymajorgrids=true,
        grid=major,
        grid style={dashed, gray!30},
        ylabel style={align=center, font=\small},
        xlabel style={align=center, font=\small},
        xticklabels={2, 8, 32, 128, 512, 2K, 8K, 32K},
        xtick={100,120,144,172,207,249,299,358},
        scaled y ticks=base 10:-6,
        ytick scale label code/.code={},
        x axis label/.style={font=\tiny},
        x tick label style={font=\tiny},
        y tick label style={font=\tiny},
        title style={font=\small},
    	yticklabel={\pgfmathprintnumber{\tick}}
}}

\pgfplotsset{mystyle16/.style={%
        width=0.35\textwidth,
        xmin=1,xmax=16,
        enlargelimits=true,
	ymajorgrids=true,
        grid=major,
        grid style={dashed, gray!30},
        ylabel style={align=center, font=\small},
        xlabel style={align=center, font=\small},
        xtick={1,2,4,8,16}, 
         x tick label style={font=\tiny},
         y tick label style={font=\tiny},
        scaled y ticks=base 10:-6,
        ytick scale label code/.code={},    
        title style={font=\small},
    	yticklabel={\pgfmathprintnumber{\tick}}
}}

\pgfplotsset{memstyle16/.style={%
        width=0.35\textwidth,
        xmin=1,xmax=16,
        ymin=0,ymax=120,
        enlargelimits=true,
	ymajorgrids=true,
        grid=major,
        grid style={dashed, gray!30},
        ylabel style={align=center, font=\small},
        xlabel style={align=center, font=\small},
        xtick={1,2,4,8,16}, 
         x tick label style={font=\tiny},
         y tick label style={font=\tiny},
%        scaled y ticks=base 10:-6,
        ytick scale label code/.code={},    
        title style={font=\small},
    	yticklabel={\pgfmathprintnumber{\tick}}
}}

\pgfplotsset{chunksStyle/.style={%
        width=0.31\textwidth,
        xmin=1,xmax=55,
        enlargelimits=true,
	ymajorgrids=true,
        ylabel style={align=center},
        ytick scale label code/.code={},
       scaled x ticks=base 10:-1,
       scaled y ticks=base 10:-3,
       xtick scale label code/.code={},
       yticklabel={\pgfmathprintnumber{\tick}}
}}

\pgfplotsset{chunksUtilStyle/.style={%
        width=0.31\textwidth,
        xmin=1,xmax=55,
        enlargelimits=true,
	ymajorgrids=true,
        ylabel style={align=center},
       scaled y ticks=base 10:-3,
       ytick scale label code/.code={},
       scaled x ticks=base 10:-1,
       xtick scale label code/.code={},
       yticklabel={\pgfmathprintnumber{\tick}}
}}

\pgfplotsset{chunkLegendStyle/.style={%
	   %width=0.35\textwidth,
      % legend columns=-1,
	 legend columns=4,
	  legend entries={total, sorted , duplicates, nulls},
	  legend style={at={(1, 0)}, anchor=north east, font=\tiny},
	   legend to name=chunksDropHalfLegend,	  
}}

\pgfplotsset{chunkCountLegendStyle/.style={%
	   %width=0.35\textwidth,
      % legend columns=-1,
      	legend columns=2,
	legend entries={random, drop-half},
	legend style={font=\tiny},
	legend to name=chunksLegend,
}}


\pgfplotsset{sPut/.style={%
	   %width=0.35\textwidth,
       legend columns=-1,
	   legend entries={\kiwi, \kary ,\skiplist, \snaptree},
	   legend to name=sPutLegend,
}}

\pgfplotsset{putScan/.style={%
	   %width=0.35\textwidth,
       legend columns=-1,
	   legend entries={\kiwi, \kary ,\skiplist, \snaptree},
	   legend to name=putScanLegend,
}}

\pgfplotsset{scansOnly/.style={%
	   %width=0.35\textwidth,
       legend columns=-1,
	   legend entries={\kiwi, \kary, \skiplist (non-atomic), \snaptree},
	   legend to name=scansOnlyLegend,
}}

\pgfplotsset{unbalanced/.style={%
	   %width=0.35\textwidth,
       legend columns=-1,
	   legend entries={\kiwi, \kary ,\skiplist, \snaptree},
	   legend to name=singleKeyOpsLegend,
}}


\pgfplotsset{balanced/.style={%
	   %width=0.35\textwidth,
       legend columns=-1,
	   legend entries={\autoTreap, \danaAVL,\snaptree, \friendly,  \stmTreap,
	   \domTreap, \optAutoTreap}, legend to name=balancedLegened,
}}

\pgfplotsset{skiplist/.style={%
	   width=\textwidth,
       ymin=0,ymax=2800000,
       xtick={2,4,8,16,32}, 
       legend columns=-1,
	   legend entries={\autoSkiplist, \kary , \skiplist (not atomic), \stmSkiplist,
	   \domSkiplist}, legend to name=skiplistLegened,
}}

\pgfplotsset{skiplist1000/.style={%
	   width=\textwidth,
       ymin=0,ymax=150000,
       scaled y ticks=base 10:-3,
       xtick={2,4,8,16,32}, 
       legend columns=-1,
	   legend entries={\autoSkiplist, \kary, \skiplist (not atomic), \stmSkiplist,
	   \domSkiplist}, legend to name=skiplistLegened1000,
}}

\pgfplotsset{skiplistUpdate/.style={%
	   width=0.35\textwidth,
       ymin=0,ymax=5500000,
       xtick={1,2,4,8,16,32}, 
       legend columns=-1,
	   legend entries={\autoSkiplist, \kary , \skiplist (not atomic), \stmSkiplist,
	   \domSkiplist}, legend to name=skiplistLegenedUpdate,
}}

\newcommand{\remove}[1]{}

\hyphenation{skip-list}

%-------------Theorem Definitions ---------------%
\newtheorem{theorem}{Theorem}[section]
\newtheorem{lemma}[theorem]{Lemma}
\newtheorem{claim}[theorem]{Claim}
\newtheorem{observation}[theorem]{Observation}
\newtheorem{proposition}[theorem]{Proposition}
\newtheorem{corollary}[theorem]{Corollary}
\newtheorem{invariant}{Invariant}

%\newenvironment{proof}[1][Proof]{\begin{trivlist}
%\item[\hskip \labelsep {\bfseries #1}]}{\qedsymb\end{trivlist}}
\newenvironment{definition}[1][Definition]{\begin{trivlist}
\item[\hskip \labelsep {\bfseries #1}]}{\end{trivlist}}
\newenvironment{example}[1][Example]{\begin{trivlist}
\item[\hskip \labelsep {\bfseries #1}]}{\end{trivlist}}
\newenvironment{remark}[1][Remark]{\begin{trivlist}
\item[\hskip \labelsep {\bfseries #1}]}{\end{trivlist}}


%\newcommand{\qed}{\nobreak \ifvmode \relax \else
%      \ifdim\lastskip<1.5em \hskip-\lastskip
%      \hskip1.5em plus0em minus0.5em \fi \nobreak
%      \vrule height0.75em width0.5em depth0.25em\fi}
%\newcommand{\qedsymb}{\hfill{\rule{2mm}{2mm}}}


\newcommand{\codesize}{\footnotesize}
%--------------------------------------------------%

\newcommand{\code}[1]{\textsf{\fontsize{9.4}{11}\selectfont #1}}
\newcommand{\codeF}[1]{\textsf{#1}}
\newcommand{\readV}{\code{read\_version}\xspace}
\newcommand{\readSet}{\code{read\_set}\xspace}
\newcommand{\writeV}{\code{write\_version}\xspace}

\newcommand{\reqI}{\textbf{LPR1}\xspace}
\newcommand{\reqII}{\textbf{LPR2}\xspace}

%---------Evaluation Macros-----------------------%
\newcommand{\autoTree}{LR-Tree\xspace}
\newcommand{\autoTreap}{LR-Treap\xspace}
\newcommand{\optAutoTree}{Opt-LR-Tree\xspace}
\newcommand{\optAutoTreap}{Opt-LR-Treap\xspace}
\newcommand{\autoSkiplist}{LR-Skiplist\xspace}
\newcommand{\danaTree}{LO-Tree\xspace}
\newcommand{\danaAVL}{LO-AVL\xspace}
\newcommand{\bronson}{snap-tree\xspace}
\newcommand{\friendly}{CF-Tree\xspace}
\newcommand{\skiplist}{Java skiplist\xspace}
\newcommand{\kary}{k-ary tree\xspace}
\newcommand{\lockfreeTree}{LF-Tree\xspace}
\newcommand{\globalTree}{Global-Tree\xspace}
\newcommand{\globalTreap}{Global-Treap\xspace}
\newcommand{\domTree}{Lock-Tree\xspace}
\newcommand{\domTreap}{Lock-Treap\xspace}
\newcommand{\stmTree}{STM-Tree\xspace}
\newcommand{\stmTreap}{STM-Treap\xspace}
\newcommand{\stmSkiplist}{STM-Skiplist\xspace}
\newcommand{\domSkiplist}{Lock-Skiplist\xspace}
\newcommand{\getOP}{\textsc{get}\xspace}


%---------Comments -----------------------%
\newcommand{\Idit}[1]{{\color{red}{[\textbf{Idit:} #1 ]}}}
\newcommand{\guy}[1]{{\color{red}{[\textbf{guy:} #1 ]}}}
\newcommand{\eshcar}[1]{{\textcolor{violet}{\{{\bf eshcar:} \em #1\}}}}
\newcommand{\anastas}[1]{{\textcolor{magenta}{\{{\bf Anastasia:} \em #1\}}}}
\newcommand{\dima}[1]{{\textcolor{magenta}{\{{\bf dima:} \em #1\}}}}


\newcommand{\cmark}{\ding{51}}%
%\newcommand{\xmark}{\emph\footnotesize{\sffamily X}}%
\newcommand{\xmark}{\footnotesize{\ding{55}}}%
\newcommand{\mycode}[1]{\texttt{#1}}
\algnewcommand{\LineComment}[1]{\Statex \ \  \(\triangleright\) {#1}}


\newcommand{\inote}[1]{}
\newcommand{\frameit}[2]{
    \begin{center}
    {\color{red}
    \framebox[3.3in][l]{
        \begin{minipage}{3in}
        \inred{#1}: {\sf\color{black}#2}
        \end{minipage}
    }\\
    }
    \end{center}
}

\newcommand{\todo}[1]{{\bf [[ TODO: #1 ]]}}
\newcommand{\comment}[1]{}

\newcommand{\kiwi}{KiWi}
\newcommand{\snaptree}{SnapTree}

\newenvironment{proof}{\paragraph{Proof}}%{\hfill$\square$}



\algdef{SE}[DOWHILE]{Do}{doWhile}{\algorithmicdo}[1]{\algorithmicwhile\ #1}%


\begin{document}

\setcopyright{none}
\acmJournal{TOPC}


\title{\kiwi: A Key-Value Map for Scalable Real-Time Analytics}


\author{Dmitry Basin}
%\affiliation{Yahoo Research, Oath, Haifa, Israel}  
\author{Edward Bortnikov}
%\affiliation{Yahoo Research, Oath, Haifa, Israel}  
\author{Anastasia Braginsky}
\affiliation{Yahoo Research, Oath, Haifa, Israel}  
\author{Guy Golan-Gueta} 
\affiliation{VMWare Research Group, Tel Aviv, Israel} 
\author{Eshcar Hillel}
\affiliation{Yahoo Research, Oath, Haifa, Israel}   
\author{Idit Keidar}
\affiliation{Technion and Yahoo Research, Oath, Haifa, Israel}   
\author{Moshe Sulamy}
\affiliation{Tel Aviv University, Tel Aviv, Israel}



\date{}


\maketitle
 \renewcommand{\shortauthors}{Basin et al.}

\begin{abstract}
Modern big data processing platforms employ huge in-memory \emph{key-value} (KV) maps.
Their applications simultaneously drive high-rate data ingestion \emph{and} large-scale analytics.
These two scenarios expect KV-map implementations that scale well with both real-time updates
and large atomic scans triggered by range queries.
%However, today's state-of-the art concurrent KV-maps  fall short of satisfying these requirements.
% -- they either provide only limited or non-atomic scans,  or severely hamper updates when scans are ongoing.

We present {\kiwi}, the first atomic KV-map to efficiently support simultaneous  large scans and real-time access.
The  key to achieving this is treating scans as first class citizens, and organizing the data structure around them.
%whereas most existing concurrent KV-maps do not provide
%atomic scans, and some others add them to existing maps without rethinking the design anew.
\kiwi\ provides {wait-free} scans,
%which is important for avoiding livelock and wasted work,
whereas its put operations are lightweight and lock-free.
It optimizes memory management jointly with  data structure access.
%We prove \kiwi's correctness and also implement it
We implement \kiwi\
and compare it to state-of-the-art solutions.
%our evaluation shows that \kiwi\ is significantly faster than the state-of-the-art.
Compared to other  KV-maps providing atomic scans,
%in workloads with long scans and concurrent writes,
\kiwi\ performs either long scans or concurrent puts an order of magnitude faster. Its   scans are twice as fast as  \emph{non-atomic} ones implemented
via iterators in the Java skiplist.


\inote{
A number of novel aspects of the {\kiwi} algorithm make it particularly well-suited for today's data processing systems.
First, it provides wait-free scans, which is important for avoiding livelock and wasted work, given that snapshot scans are often long.
On the other hand, given that updates are short, \kiwi\ opts to make them lock-free rather than wait-free, since restarts due to conflicts
are practically rare and waste little work.  Second, {\kiwi} uses multi-versioning, but keeps old data versions selectively,
only as needed for ongoing scans, and furthermore keeps the updates light-weight by deferring version management to scans.
Third, the {\kiwi} algorithm optimizes memory management and reclamation jointly with the optimization of data structure access.
Finally, {\kiwi} implements a balanced data structure, which allows operations to run orders of magnitude faster than
unbalanced ones in case update order is not random.
}

\end{abstract}


\renewcommand{\thefootnote}{\arabic{footnote}}

\section{Introduction}
\label{sec:intro}

{\bf{Motivation and goal.}} The ordered \emph{key-value} (KV) map abstraction has been recognized as a popular programming interface
since the dawn of computer science, and remains an essential component of virtually any computing system today.
It is not surprising, therefore, that with the advent of multi-core computing,  many scalable concurrent
implementations have emerged,
e.g.,~\cite{JavaConcurrentSkipList,LinkedListBP,BraginskyP2012,Hendler04,Kogan12,NatarajanM2014}.

KV-maps have become centerpiece to web-scale data processing systems such as
Google's F1~\cite{Shute2013}, which powers its AdWords
%\footnote{\url{https://www.google.com/adwords/}}
business, and Yahoo's Flurry~\cite{flurry} --
%\footnote{\url{https://developer.yahoo.com/flurry/docs/analytics/}},
the technology behind Mobile Developer Analytics.
For example,
as of early 2016,  Flurry  reported systematically collecting data of 830\!,000 mobile
apps~\cite{appmatrix}
%\footnote{\url{http://flurrymobile.tumblr.com/post/144245637325/}}  % appmatrix
running on 1.6 billion
user devices~\cite{phablet}.
%\footnote{\url{http://flurrymobile.tumblr.com/post/117769261810/}}. % the-phablet-revolution
Flurry streams  this data into a massive index, and provides
%application developers with tools that produce
a wealth of reports over the collected data.
Such
\emph{real-time analytics}  applications push KV-store scalability requirements to new levels and raise novel use cases.
Namely, they require
%  real-time performance for
both
(1) low latency ingestion of incoming data, and (2) high performance analytics of the resulting dataset.

The stream scenario requires the KV-map to support fast \emph{put} operations,
whereas  analytics  relies on (typically large) \emph{scans} (i.e., range queries).
The consistency (atomicity) of scans is essential for correct analytics.
The new challenge that arises in this environment is allowing consistent scans
to be obtained \emph{while the data is being updated in real-time}.

% KiWi is about balancing scans and updates
We present {\kiwi}, the first KV-map to efficiently support large atomic
scans as required for data analytics alongside real-time updates.
Most concurrent KV-maps today do not support atomic scans at all~\cite{JavaConcurrentSkipList,LinkedListBP,BraginskyP2012,Hendler04,NatarajanM2014,Kogan12,Lomet13,ArbelGHK15}.
A handful of recent works support atomic scans in KV-maps, but they either
hamper updates when scans are ongoing~\cite{BronsonCCO2010,Prokopec12},
or fail to ensure progress to scans in the presence of updates~\cite{BrownA12}.
See Section~\ref{sec:related} for a discussion of related work.

The emphasis in \kiwi's design is on facilitating synchronization between 
scans and updates. Since scans are typically long, our solution avoids livelock 
and wasted work by always allowing them to complete (without ever needing to 
restart). On the other hand, updates are short (since only single-key puts are supported), 
therefore restarting them in cases of conflicts is practically ``good enough'' -- 
restarts are rare, and when they do occur, little work is wasted. 
Formally, \kiwi\  provides \emph{wait-free} gets and scans and \emph{lock-free} puts.


{\bf{Design principles.}}
% Versions on-demand, managed by scans
To support atomic wait-free scans, \kiwi\ employs multi-versioning~\cite{BHG:Book87}. But in contrast to the standard approach~\cite{mv-stm-chapter}, where each put creates a new version for the updated key, \kiwi\ only keeps old versions that are needed for ongoing scans, and otherwise overwrites the existing version. Moreover, version numbers are managed by scans rather than updates, and put operations may
overwrite data without changing its version number. This unorthodox approach offers significant performance gains given that scans typically retrieve large amounts of data and hence take much longer than  updates.
It also necessitates a fresh approach to synchronizing updates and scans, which is a staple of \kiwi's design.

% Chunks
A second important consideration is efficient memory access and management. Data in \kiwi\ is organized
as a collection of {\em chunks}, which are large blocks of contiguous key ranges.
Such data layout is cache-friendly and suitable for non-uniform memory architectures (NUMA), as it
allows long scans to proceed with few fetches of new data to cache or to local memory.
Chunks regularly undergo maintenance to improve their internal organization and space utilization (via \emph{compaction}), and the distribution of key ranges into chunks (via splits and merges).
% All these issues  are handled by
\kiwi's {\em rebalance\/} abstraction performs batch processing
of such maintenance operations. The synchronization of
rebalance operations with ongoing puts and scans is subtle, and much of the \kiwi\ algorithm is dedicated to handling
possible races in this context.

% Index
Third, to facilitate concurrency control, we separate chunk management from indexing for fast lookup:
\kiwi\ employs an \emph{index} separately from the (chunk-based) data layer.
The index is updated lazily once rebalancing of the data layer completes.

% Data structure Balanced
Finally, \kiwi\ is a balanced data strucutre, providing logarithmic access latency in the absence of contention.
This is achieved via a combination of (1) using a balanced index for fast chunk lookup and (2) partially sorting keys in each
chunk to allow for fast in-chunk binary search. The \kiwi\ algorithm is detailed in Section~\ref{sec:alg}
and we prove its correctness in Section~\ref{sec:proof}.
% \todo{Consider saying we're the only data structure snapshot algorithm that has a formal proof, which is in the appendix.}


{\bf{Evaluation results.}}
\kiwi's Java implementation is available in github\footnote{https://github.com/sdimbsn/KiWi}. In Section~\ref{sec:eval} we benchmark it under multiple representative workloads.
In the vast majority of  experiments, it significantly outperforms existing concurrent KV-maps that support scans.
\kiwi's advantages are particularly pronounced in our target scenario with long scans in the presence of concurrent puts, where
it not only performs \emph{all} operations faster than the competitors~\cite{BrownA12,BronsonCCO2010},
but actually executes either updates or scans an order of magnitude faster than every other solution supporting atomic scans.
Notably, \kiwi's atomic scans are also two times faster than the \emph{non-atomic}
ones offered by the Java Skiplist~\cite{JavaConcurrentSkipList}. 




\section{Related Work}
\label{sec:related}

\paragraph{Techniques.}
\kiwi\ employs a host of techniques for efficient synchronization, many of which have been used in similar contexts before.
Multi-versioning~\cite{BHG:Book87}
%\todo{cite some DB source}
is a classical database approach for allowing atomic scans in the presence of updates,
and has also been used  in the context of transactional memory~\cite{mv-stm-chapter}. 
In contrast to standard multi-versioning, \kiwi\ does not create a new version for each update, 
and leaves version numbering to scans rather than updates.

Braginsky and Petrank used lock-free chunks  for efficient memory management
in the context of non-blocking linked lists~\cite{LinkedListBP} and B$^{+}$trees~\cite{BraginskyP2012}.
However, these data structures do not support atomic scans as \kiwi\ does.
\inote{
%% Idit: omitted below, does not compare the current work to previous work
There, list elements are grouped into chunks for better cache locality and list traversal performance. The chunks maintain the predetermined size boundaries.
The \emph{freeze} and \emph{restructure} lock-free techniques are used in \cite{LinkedListBP} for chunks exchange,
where freezing makes a chunk immutable and notifies threads that the part of the data structure they are currently using is obsolete.
%The similar lock-free chunk mechanism was later used by the same authors for building lock-free, balanced, B+tree \cite{BraginskyP2012}.
}

\kiwi\ separates index maintenance from the core data store, based  on the observation that index updates are only needed for efficiency and not for correctness, and hence can be done lazily. This observation was previously leveraged, e.g.,
for a concurrent skip list, where only the underlying linked list is updated as part of the atomic operation and other links are updated lazily~\cite{bpc16,HerlihyLLS2007,HerlihyS2008,Spiegelman:2016}.
%For providing wait-free scans, we rely on the standard approach of helping~\cite{}, which has been used extensively in the past.

\inote{
%% Idit: omitted below, irrelevant
Snapshot \cite{Afek93} is a possible approach for implementing range queries. The majority of snapshot algorithms, e.g. \cite{Fatourou07,Jayanti05}, have theoretical impact.
Their API is limited (no read), and it is hard to use them to efficiently implement non-blocking range queries.
}


\begin{table*}[ht]
\codesize
\begin{center}
\begin{tabular}{|l|cccc:cc|}

  \hline
  {\bfseries } & \multicolumn{4}{c:}{\bfseries scans}  & \multicolumn{2}{c|}{\bfseries performance} \\
  {\bfseries } & {\bfseries atomic} & {\bfseries multiple } & {\bfseries partial} & {\bfseries wait-free } & {\bfseries  balanced} & {\bfseries fast puts} \\
%  \hline\hline
\hline

   Ctrie \cite{Prokopec12}                      & $\checkmark$ & $\checkmark$ & \xmark              & \xmark           & $\checkmark$ & \xmark \\

  \snaptree\ \cite{BronsonCCO2010}    & $\checkmark$ & $\checkmark$ &$\checkmark$   & \xmark            & $\checkmark$ & \xmark \\

  \kary\ \cite{BrownA12}                       & $\checkmark$  & $\checkmark$ & $\checkmark$ & \xmark & \xmark  &$\checkmark$ \\

%  BW-Tree \cite{Lomet13}  & \xmark & $\checkmark$ & $\checkmark$ & \xmark            & $\checkmark$ \\
 % \hline

  {snapshot iterator \cite{Petrank2013}} & $\checkmark$ & \xmark           & \xmark & $\checkmark$ & $\checkmark$ & $\checkmark$ \\

\skiplist\ \cite{JavaConcurrentSkipList} & \xmark          & $\checkmark$ & $\checkmark$ & $\checkmark$ & $\checkmark$ & $\checkmark$ \\

 % \hline
{\bfseries \kiwi} & \ding{52} & \ding{52} & \ding{52} & \ding{52} & \ding{52} & \ding{52} \\
  \hline
\end{tabular}
\end{center}
\caption{Comparison of concurrent data structures implementing scans. For range queries, support for multiple partial scans is necessary.
Fast puts do not hamper updates (e.g., by cloning nodes) when scans are ongoing.}
\label{tab:overview}
\end {table*}

\paragraph{Concurrent maps supporting scans.}
Table~\ref{tab:overview} summarizes the properties of  state-of-the-art concurrent data structures that support scans, and compares them to {\kiwi}.
\snaptree~\cite{BronsonCCO2010} and Ctrie  \cite{Prokopec12} use lazy copy-on-write for cloning the  data structure in order to
support snapshots.
This approach severely hampers put operations when scans are onging,
as confirmed by our empirical results for \snaptree, which was shown to outperform Ctrie. Moreover,
in Ctrie,  partial snapshots cannot be obtained.



\inote{
\snaptree~\cite{BronsonCCO2010} is a binary search tree using lazy copy-on-write to support snapshots. It
uses locks to clone the entire data structure to support range queries. This approach does not provide progress guarantees,
and may result in excessive duplication of the structure. Moreover, with this approach, put operations are severly hampered by
the support for snapshots, as confirmed by our empirical results below.
Ctrie  \cite{Prokopec12}  is a non-blocking concurrent hash trie that also uses a lazy copy-on-write scheme
that delays puts in the presence of scans. However,
in Ctrie, keys are ordered by their hashes, and so scans do not offer range queries on the original key space.
%is hard to efficiently implement range queries.
%%To do so, one must iterate over all keys in the snapshot.
}

Brown and Avni \cite{BrownA12} introduced range queries for the \kary search tree \cite{kary}.
Their scans are atomic and lock-free, and outperform those  of Ctrie and \snaptree\ in most scenarios.
However, each conflicting put restarts the scan,
degrading performance as scan sizes increase. Additionally,
\kary is unabalnced, and so its performance plunges when keys are inserted in sequential order
(a common practical use case).

Some techniques offer generic add-ons to atomic \emph{snapshot iterator} in existing data structures~\cite{Petrank2013, wttm2016}.
However,~\cite{Petrank2013} supports only one scan at a time, and~\cite{wttm2016}'s throughput is lower than \kary's under low contention. 

Most concurrent key-value maps do not support atomic  scans in any
way~\cite{JavaConcurrentSkipList,LinkedListBP,BraginskyP2012,Hendler04,NatarajanM2014,Kogan12,Lomet13}.
Standard iterators implemented on such data structures provide non-atomic scans. Among these, we compare \kiwi\ to the standard Java 
concurrent skip-list~\cite{Fraser04}.
%written by Doug Lea based on \cite{Fraser04} and released as part of the JavaTM SE 6 platform.

Analytics platforms often exploit persistent KV-stores like Google's Bigtable~\cite{Chang2008},
Apache HBase~\cite{ApacheHBase}, and others~\cite{leveldb, RocksDB}. 
These technologies combine on-disk indices for persistence with an in-memory KV-map for real-time data acquisition.
%The latter's scalability has a major impact on overall system performance, (as shown åin~\cite{GolanGueta2015}).
They often support atomic scans, which can be non-blocking as long as they can be served from RAM~\cite{ GolanGueta2015}. 
However, storage access is a principal consideration in such systems, which makes their design quite different from that 
of in-memory stores as discussed in this paper.




\section{{\kiwi} Algorithm}
\label{sec:algo}
We now present the details of the {\kiwi} algorithm.
Section~\ref{sec:algo:putget} presents the basic API supporting only get and put operations. Then Section~\ref{sec:algo:scan} extends the functionality with scans, this extension builds on multi-versioning. The auxiliary pseudo-code (in Appendix~\ref{app:code}) is written in the context of the full API and therefore handles versions. All version references and accesses (marked with  blue color bold fonts) can be ignored while reading Section~\ref{sec:algo:putget}.


\subsection{Put and Get Operations}
\label{sec:algo:putget}
%This section describes the basic algorithm of put and get operations, as well as the structural rebalancing building block. 

%%% index
We assume the index is a linearizable key-chunk map data structure supporting wait-free \texttt{find} method and lock-free \texttt{replace} method.
Many list-based data structure (such as skip-list and $B^+$-tree) implementations~\cite{Fraser04, HerlihyLLS2007, BraginskyP2012,ArbelA2014,DrachslerVY2014} can serve as basis for supporting this functionality. 
\begin{description}
\setlength{\itemsep}{0pt}
\setlength{\parskip}{0pt}
\item[\texttt{find($k$)}:] returns the maximal chunk with minimal-key smaller or equal to $k$; 
\item[\texttt{replace($oldChunks$, $newChunks$)}:] replaces the mapping to $oldChunks$ with mapping to $newChunks$; ensures each new chunk is added to the index at most once\footnote{Can be supported by augmenting the next pointer of each key in the ``leafs''-list with \emph{aba counters}, as in \cite{BraginskyP2012}. 

\noindent 
ABA problem: thread t reads value
A from l (next pointer in our case), then other thread changes l to B and then back to A, later t successfuly
applies CAS on l whereas it should have failed.
}.
\end{description}



%%% get operation
{\kiwi} \emph{get} operation (see Figure~\ref{fig:get}) returns the value associated with the passed key, $k$, or $\bot$ if $k$ is not in the map.
The operation uses the index and then traverses the chunk list to find the chunk spanning $k$.
Within this chunk the operation seeks $k$ in the cell linked-list. It starts with a binary search in the batched prefix at the head of the list. Once the closest key is found in the prefix the search follows the links in the cell list to find $k$, and returns the value associated with it. 

\begin{figure*}
\codesize
	\begin{center}
	\begin{subfigure}[b]{.47\textwidth}
		\begin{algorithmic}[1]{}
		\Function{kiwi::get}{Key key}
		\State curr = findChunk(key) 
		\State return curr.find(key)
		\EndFunction
		\vspace{2mm}
		\Function{kiwi::findChunk}{Key key}
		\State curr = index.find(key) \Comment{curr minimal key $\leq$ key}
		\State next = curr.next()
		\State \textbf{while} next$\neq$null and next.getMinKey() $\leq$ key \textbf{do} 
		\State \ \  curr = next
		\State \ \ next = curr.next()
		\State return curr
		\EndFunction
		\end{algorithmic}
		\caption{get
%First find a chunk in the index, then traverses the chunk list to find the maximal chunk with minimal-key less or equal to $key$. Finally, return the value associated with the key in the chunk or $\bot$ if $key$ is not in the map.
} \label{fig:get}
	\end{subfigure}
\hspace{0.03\textwidth}
	\begin{subfigure}[b]{.47\textwidth}
		\begin{algorithmic}[1]{}
		\Function{kiwi::put}{Key key, Value val}
		\State \textbf{while} true \textbf{do}
		\State \ \ chunk = findChunk(key) 
		\State \ \ \textbf{if} chunk is \textsc{infant} \textbf{then} \Comment{incomplete rebalance}
		\State \ \ \ \ rebalance(chunk.getParent()) \Comment{help complete}
		\State \ \ cell = chunk.allocate(key, val) 
		\State \ \ \textbf{if} cell $\neq$ null \textbf{then} 
		\State \ \ \ \ chunk.insertCell(cell) \Comment{insert cell into cell list}
		\State \ \ \ \ return
		\State \ \ rebalance(chunk) \Comment{allocation failed}
		\EndFunction
		\end{algorithmic}
		\caption{put 
%Find the chunk spanning the key; allocate space for the new cell and insert it into the cell list. If allocation failed, the chunk is \emph{frozen} or is still \emph{infant} help complete the rebalance operation. While the put is ongoing it is published in the pending put array so it can be observed by other get and rebalance operations.
} \label{fig:put}
	\end{subfigure}
	\end{center}
	\caption{{\bf {\kiwi} get and put operations}.
			\label{fig:getput}}
\end{figure*}

%%% put
{\kiwi} \emph{put} operation (see Figure~\ref{fig:put}) finds the chunk $c$ which spans the passed key similar to get operation. If $c$ cannot yet absorb the update (\textsc{infant} chunk), the put helps complete the rebalance operation on $c$'s parent.
Then the put allocates a key-value cell in the chunk to store the new data by increasing the counters pointing to the next free space in the underlying arrays.  
If the allocation is successful the put is guaranteed to finish without restarting -- the operation inserts the cell into the cell list and completes. If the allocation fails then either the chunk is full or the chunk is in the process of being rebalanced.  Either way, after helping rebalance the chunk the operation restarts.
 
The cell linked-list stores only one cell per key. Therefore, when inserting a cell into the cell list there are three options: (1) the key already exists in the linked-list and has the most recent value -- the operation does nothing (seen as it was overridden by the more recent put operation); (2) the key already exists in the linked-list but with an older value -- the value is updated in-place; (3) the key is absent -- the cell is linked into the list at the right position. 

\remove{
\begin{figure}
	\begin{center}
		\begin{algorithmic}[1]{}
		\Function{kiwi::put}{Key key, Value val}
		\State \textbf{while} true \textbf{do}
		\State \ \ chunk = findChunk(key) 
		\State \ \ \textbf{if} chunk is \textsc{infant} \textbf{then} \Comment{incomplete rebalance}
		\State \ \ \ \ rebalance(chunk.getParent()) \Comment{help complete}
		\State \ \ cell = chunk.allocate(key, val) 
		\State \ \ \textbf{if} cell $\neq$ null \textbf{then} 
		\State \ \ \ \ chunk.insertCell(cell) \Comment{insert cell into cell list}
		\State \ \ \ \ return
		\State \ \ rebalance(chunk) \Comment{allocation failed}
		\EndFunction
		\end{algorithmic}
		\caption{{\kiwi} put operation. 
%Find the chunk spanning the key; allocate space for the new cell and insert it into the cell list. If allocation failed, the chunk is \emph{frozen} or is still \emph{infant} help complete the rebalance operation. While the put is ongoing it is published in the pending put array so it can be observed by other get and rebalance operations.
} \label{fig:put}
	\end{center}
\end{figure}
}

%%% rebalance operation
A \emph{rebalance} operation (see Figure~\ref{fig:rebalance}) is responsible for keeping the chunks evenly loaded, and the index well balanced. It gets a chunk as an input, it tries to engage and rebalance this chunk's ``neighborhood''; the result is a list of balanced chunks. Split and merge operations are special cases of a rebalance operation: a split engages a single chunk and outputs a list of two chunks, a merge engages two consecutive chunks and outputs a single chunk. Appendix~\ref{app:code} further elaborates this algorithm.

Once a chunk is engaged in a rebalance it cannot be engaged with another rebalance operation. 
An important invariant is that balanced chunks store the same data as the engaged frozen chunks, only that it is more evenly distributed between the chunks and also stored in consecutive locations in the batch prefix (which later allows utilizing binary search). All undeleted non-obsolete cells are copied by their order in the cell lists into new chunks; each chunk is written until it is half full. 

To complete the rebalance operation the engaged chunks are replaced with the new balanced chunks.
If the link from chunk $c$, previous to the first engaged chunk, is marked as deleted then $c$ is engaged in another rebalance and the current rebalance operation recursively helps it to complete. There can be a scenario in which the current operation helps other rebalance operation infinitely. While it means that for each such helping round many put operations have made progress, this implies that the rebalance operation, and therefore the put operation is lock-free but not wait-free.  

%%% pending array - why it is needed and how it is used
A data may be lost if put operations continue to add cells to a chunk while a rebalance operation is traversing the cell list and copies its content to a new chunk. 
We remedy this problem by tracking put operations  
in a designated \emph{pending put array} (PPA), with a per thread entry. A put is published in the chunk's PPA so it becomes visible to other get and rebalance operations; when the operation completes, the entry is cleared. A rebalance operation \emph{freezes} all entries in the pending set thereby blocking any additional cells from being inserted into the chunk's cell list. 
The rebalance operation helps to complete all put operations that are already published in the PPA. Get operations also go through the PPA to find the most recent value, specifically the value which was allocated most recently. 

%For lack of space, the 
Auxiliary code for get, put and rebalance appear in Appendix~\ref{app:code} in Figure~\ref{fig:aux}.%Figures~\ref{fig:find},~\ref{fig:allocate} and~\ref{fig:freeze}, respectively.


%\eshcar{Add a paragraph about rebalance triggers}

\subsection{Supporting Scans}
\label{sec:algo:scan}

%describe how we support versions - changes in put/get/rebalance operations. scan operations. pseudo-code.
We implement linearizable scans using the common approach of multi-versioning: each key-value cell is stored in the chunk together with a unique, monotonically increasing, \emph{version}. The versions are internal, and are not exposed to {\kiwi} users. 
%To support multi-versioning {\kiwi} maintains a global counter, \emph{globalVersion}. Scans increase the counter; put operations read the counter and install key-version-value tuples in the chunks cells. 
Figures~\ref{fig:find},~\ref{fig:allocate} in Appendix~\ref{app:code} highlight the changes required to support versions.

This means each chunk potentially holds many versions for a given key. Following standard practice old versions are not removed from the chunk i.e., they exist at least until the chunk is discarded following a rebalance operation. Obsolete versions are removed while rebalancing the chunk once they are no longer needed for any scan. In other words, for every key and every scan, the latest version of the key that does not exceed the scan's read point is kept.

A scan operation (see Figure~\ref{fig:scan}) gets as input a minimum key and a maximum key. It begins by acquiring a \emph{read point}; this is done by atomically incrementing and reading the global version counter. Then the scan iterates over the chunk list starting from the chunk which spans the minimum key. At each chunk, the operation iterates over the pending list and the cell list, and filters out cells that do not belong to it: for each key k in the range, the scan filters out cells that have higher version than the scan read point, or are older than the latest version (of key k) that does not exceed the scan read point. As in the basic algorithm, we break ties by comparing the locations of the values in the chunk; the value that was allocated later is considered more recent.

To consolidate with the rebalancing operation, the scan installs its read point in a global list that captures all \emph{active scans}. Rebalance operations query the list to identify the maximal version before which versions can be removed. The scan cannot increase the global counter and its entry in the scan array atomically. Therefore, to avoid a potential race between installing a scan read point and it being observed by a rebalance operation, we employ \emph{helping}. The scan publishes an \emph{empty entry} in the scan array and then tries to set it with the version it acquired. Concurrent rebalance operations help scans install a version in empty entries; 
monotonically increasing \emph{aba} counters
%\footnote{ABA problem: thread t reads value A (empty entry in our case) from l, then other thread changes l to B and then back to A, later t successfuly applies CAS on l whereas it should have failed.} 
ensure a rebalance only helps scans it must help.
The version that is written first is the one that the scan uses.


\begin{figure*}
\codesize
	\begin{center}
	\begin{subfigure}[t]{.48\textwidth}
		\begin{algorithmic}[1]{}
		\Function{kiwi::rebalance}{Chunk chunk}
		\State rebalanceObj = buildRebalanceObj(chunk)
		\State engaged = rebalanceObj.engageChunks()
		\State \textbf{forall} engChunk in engaged \textbf{do}
		\State \ \ engChunk.freeze()
		\State engaged.last().markDeleted() 
		\State balanced = chunk.getRebalanceObj().balance()
		\State replace(engaged, balanced)
		\State \textbf{for all} chunk in balanced \textbf{do}
		\State \ \ chunk.makeNonInfant()
		\EndFunction
		\vspace{2mm}
		\Function{kiwi::replace}{List engaged, List balanced}
		\State next = engaged.last().next()
%		\State \textbf{if} balanced.isNotEmpty() \textbf{then}
%		\Statex \ \  \Comment{connect balanced tail}
%		\State \ \ CAS(balanced.last().nextPtr, $\langle null,0\rangle$, $\langle next,0\rangle$)
		\State firstEng = engaged.first()
		\State \textbf{if} balanced.isNotEmpty() \textbf{then}
		\State \ \ next = balanced.first()
		\State \textbf{while} true \textbf{do}
		\State \ \ prev = findPrevInChunkList(firstEng)
		\State \ \ \textbf{if} prev == $\bot$ \textbf{then} break \Comment{done replacement}
		\State \ \ \textbf{if} prev.nextPtr.mark$\neq 0$ \textbf{then} \Comment{marked deleted}
		\State \ \ \ \ rebalance(prev) \Comment{help rebalance prev}
		\State \ \ \textbf{else}  \Comment{connect balanced head}
		\State \ \ \ \ CAS(prev.nextPtr, $\langle firstEng,0\rangle$, $\langle next,0\rangle$) 
		\State index.replace(engaged, balanced) 
		\EndFunction
		\end{algorithmic}
		\caption{rebalance
%Done in 4 stages: (1)~engage neighborhood, (2)~freeze all engaged chunks, (3)~copy undeleted cells to balanced chunks, (4)~replace engaged chunks with balanced chunks both in chunk list and in index. Finally, to make the balanced chunks available for new updates they are marked as non-infant.
} \label{fig:rebalance}
	\end{subfigure}
\hspace{0.03\textwidth}
	\begin{subfigure}[t]{.46\textwidth}
		\begin{algorithmic}[1]{}
		\Function{kiwi::scan}{Key minKey, Key maxKey}
		\State aba = s[\textit{tid}].aba
		\State s[\textit{tid}] = $\langle EMPTY,aba\rangle$ \Comment{publish in scan array}
		\State ver = FAI(globalVersion) \Comment{acquire read point}
		\State CAS(s[\textit{tid}], $\langle EMPTY,aba\rangle$, $\langle ver,aba\rangle$) 
		\State ver = s[\textit{tid}]
		\State chunk = findChunk(key) 
%		\Statex \Comment{finds first cell with key $\geq$ minKey and version $\leq$ ver}
		\State cell = chunk.findCell(minKey, ver) 
		\State key = cell.key
		\State \textbf{while} key $\leq$ maxKey \textbf{do} 
		\State \ \ pending = chunk.findInPending(key,maxKey)  
		\State \ \ \textbf{while} key $\neq$ null and key $\leq$ maxKey \textbf{do}
		\State \ \ \ \ pos = $\max$(cell.val, pending.get(key))
		\State \ \ \ \ res.append(chunk.v[pos]) \Comment{build result}
		\State \ \ \ \ cell = chunk.findNextCell(cell, ver)
		\State \ \ \ \ \textbf{if} cell == null \textbf{then} key = null
		\State \ \ \ \ \textbf{else} key = cell.key %\Comment{end of cell list loop}
		\Statex \Comment{end of cell list loop in chunk}
		\State \ \ chunk = chunk.next()
		\State \ \ cell = chunk.firstCell()
		\State \ \ key = cell.key \Comment{end of chunks list loop}
%		\Statex \Comment{end of chunks list loop}
		\State s[\textit{tid}] = $\langle null,aba+1\rangle$ \Comment{unpublish scan}
		\State return res
		\EndFunction
		\end{algorithmic}
		\caption{scan - v is the chunk's values array, s is the global scan array, \textit{tid} is the thread id
%The scan acquires a read point, and starts traversing the range ($minKey$-$maxKey$) chunk by chunk. For each key in the range, collects the most recent value before the read point. Finds the values by seeking both pending put array and the cell list, similar to get. Concurrent rebalance operations keep relevant versions, as the read point is visible in the scan array.
} \label{fig:scan}
	\end{subfigure}
	\end{center}
	\caption{{\bf {\kiwi} rebalance and scan operations}.
			\label{fig:rebalscan}}
\end{figure*}

\remove{
\begin{figure}[t]
	\begin{center}
		\begin{algorithmic}[1]{}
		\Function{kiwi::scan}{Key minKey, Key maxKey}
		\State s[\textit{tid}] = EMPTY \Comment{publish in scan array}
		\State ver = FAI(globalVersion)
		\State CAS(s[\textit{tid}], EMPTY, ver) \Comment{agree on read point}
		\State ver = s[\textit{tid}]
		\State chunk = findChunk(key) 
		\State next = chunk.next()
%		\Statex \Comment{finds first cell with key $\geq$ minKey and version $\leq$ ver}
		\State cell = chunk.findCell(minKey, ver) 
		\State key = cell.key
		\State \textbf{while} key $\leq$ maxKey \textbf{do} 
		\State \ \ pending = chunk.findInPending(key,maxKey)  
		\State \ \ \textbf{while} key $\neq$ null and key $\leq$ maxKey \textbf{do}
		\State \ \ \ \ pos = $\max$(cell.val, pending.get(key))
		\State \ \ \ \ res.append(v[pos]) \Comment{build result}
		\State \ \ \ \ cell = chunk.findNextCell(cell, ver)
		\State \ \ \ \ \textbf{if} cell == null \textbf{then} key = null
		\State \ \ \ \ \textbf{else} key = cell.key
		\Statex \Comment{end of cell list loop in chunk}
		\State \ \ chunk = chunk.next()
		\State \ \ cell = chunk.firstCell()
		\State \ \ key = cell.key 
		\Statex \Comment{end of chunks list loop}
		\State s[\textit{tid}] = null \Comment{unpublish scan}
		\State return res
		\EndFunction
		\end{algorithmic}
		\caption{{\kiwi} scan operation. 
%The scan acquires a read point, and starts traversing the range ($minKey$-$maxKey$) chunk by chunk. For each key in the range, collects the most recent value before the read point. Finds the values by seeking both pending put array and the cell list, similar to get. Concurrent rebalance operations keep relevant versions, as the read point is visible in the scan array.
} \label{fig:scan}
	\end{center}
\end{figure}
}
 
\remove{
We assume chunk index is a linearizable map data structure supporting the lock-free \texttt{put} and \texttt{delete} methods and the wait-free \texttt{find} method, as for example in ~\cite{HerlihyLLS2007}. Additionally, we assume lock-free implementation of \texttt{replace} and \texttt{remove} methods, as can be found in the
\texttt{ConcurrentSkipListMap}, written by Doug Lea based on work by Fraser and Harris \cite{Fraser04}. Finally, the chunk index is required to support \texttt{loadLink} and \texttt{storeConditional} methods as explained below. Those methods can be added to index using version numbers on the key's next pointers as it was done in \cite{BraginskyP2012}. Hereby, we specify the required interface of the key to chunk index: 
\begin{description}
\item[\texttt{put($k$, chunk)}:] adds a mapping from $k$ to the \texttt{chunk}; 
\item[\texttt{delete($k$)}:] removes the key $k$; 
\item[\texttt{find($k$)}:] returns the chunk with minimal key smaller or equal to $k$; 
\item[\texttt{replace($k$, oldChunk, newChunk)}:] if the mapping from $k$ to the \texttt{oldChunk} exists, it is atomically replaced with a mapping from $k$ to the \texttt{newChunk};
\item[\texttt{remove($k$, chunk)}:] if the mapping from $k$ to the \texttt{chunk} exists, the key $k$ is atomically removed; 
\item[\texttt{mark loadLink($k$)}:] returns a mark, according to which it will be possible to distinguish whether any key was inserted in between $k$ and a key next to $k$ after last \texttt{loadLink} call;   
\item[\texttt{storeConditional($k$, chunk, mark)}:] if no key was inserted in between $k$ and a key next to $k$ since last \texttt{loadLink}, the mapping from $k$ to the \texttt{chunk} is atomically added
\end{description}
}

\newcommand{\lp}[1]{LP(\ensuremath{#1})}

\section{Correctness}
\label{sec:proof}

In order to lay out foundations  for reasoning about correctness,  we  define in Section~\ref{sec:spec} the model and correctness notion we seek to prove. 
We proceed to prove the algorithm's safety in Section~\ref{sec:safe} and liveness in Section~\ref{sec:live}.

\subsection{Model and Correctness Specification}
\label{sec:spec}

We consider an asynchronous shared memory model~\cite{Welch2004} consisting of a collection of shared variables accessed by a finite number of threads , which also have local state.
High-level objects, such as a map, are implemented using low-level memory objects supporting atomic read, write, and read-modify-write (e.g., CAS) primitives. 
Threads  \emph{invoke} high-level \emph{operations}, which perform a sequence of  \emph{steps} on low-level objects, and finally \emph{return}.

An \emph{algorithm} defines the behaviors of threads executing high-level operations as deterministic state machines, where local state transitions are associated with  shared low-level memory 
accesses (read, write, CAS, etc.) or high-level invocations/responses.
A \emph{configuration} describes the  local states of all threads and the contents of shared variables. An \emph{initial configuration} is one where all threads and variables  are in their initial values.
An \emph{execution} of algorithm $\mathcal{A}$ is an alternating sequence of configurations and steps, beginning with some initial configuration, 
such that configuration transitions occur according to $\mathcal{A}$.
Operation $op1$ \emph{precedes} operation $op2$ in an execution if $op1$'s return step precedes $op2$'s invoke step;
two operations are \emph{concurrent} in execution $\sigma$  if neither precedes the other, that is, both are invoked in $\sigma$  before either returns.
 In a  \emph{sequential} execution, there are no concurrent operations.
We use the notion of time \emph{t} during an execution $\sigma$  to refer to the configuration reached after the $t^{th}$ step in $\sigma$.
An \emph{interval} of execution $\sigma$ is a sub-sequence of $\sigma$.
The \emph{interval of an operation} $op$ in $\sigma$  starts with the invocation step of $op$ and ends with the configuration following the return from $op$ or 
the end of $\sigma$, if there is no such return.

Our correctness notion is \emph{linearizability}, which intuitively means that the object ``appears to be'' executing sequentially. 
More formally, the \emph{history} $H(\sigma)$ of execution $\sigma$ is the sequence of invocations  and returns occurring in $\sigma$. 
In a sequential history, each invocation is immediately followed by its return. 
An object is specified using a \emph{sequential specification}, which is the set of its allowed sequential histories. 

For a history $h$, \emph{complete($h$)} is the sequence obtained by removing invocations with no responses from $h$.
We assume that histories are \emph{well-formed}, meaning that the sub-sequence of each thread's steps in a history is sequential.
An algorithm is \emph{linearizable}~\cite{HerlihyW1990} if each of its histories $h=H(\sigma)$ can be extended by adding zero or more response events to a history $h'$, 
so that  \emph{complete($h'$)} has a sequential permutation that preserves $h$'s precedence relation and satisfies the object's sequential specification. 
Thus, a linearizable algorithm provides the illusion that each invoked operation takes effect instantaneously at some  \emph{linearization point} inside its interval. 

For liveness, we consider two notions: \emph{wait-freedom} requires that \emph{every} operation return within a finite number of its own steps, whereas \emph{lock-freedom}  
requires only that \emph{some} operation return within a finite number of steps. The former is sometimes called \emph{starvation-freedom} and the latter -- \emph{non-blocking}. 

\kiwi\ implements a linearizable map offering lock-free put operations and wait-free get and scan operations. 
In its sequential specification, get and scan return the latest value inserted by a put for each key in their ranges.





\subsection{\kiwi's Linearizability}
\label{sec:safe}

In Section~\ref{ssec:rebalance-proof} we show that the rebalance process preserves the data structure's integrity and contents. 
We then prove that  {\kiwi} is linearizable by identifying, for every operation in a given execution, a {linearization point} between its invoke and return steps, so that the operation ``appears to'' occur atomically at this point.  We discuss put operations in Section~\ref{ssec:put-proof}, and gets and scans in Section~\ref{ssec:get-proof}. 
The linearization point of operation $op$ is denoted \lp{op}. 



\subsubsection{Rebalance.}
\label{ssec:rebalance-proof}

We first argue that rebalance operations preserve the integrity of the data structure.  
To this end, we introduce some definitions. 
We say that a chunk $C$ is \emph{accessible} in \kiwi\ if $C$ is connected to the linked list, 
that is, if traversing the lined list from its head to its tail goes through $C$. 
While a chunk is accessible its key range is well-defined: 
We say that key $k$ is in the \emph{range} of chunk $C$ if $k \geq C$.\code{minKey} and $k < C.$\code{next.minKey}.

When all the entries in a chunk's PPA are frozen, we say that the chunk is \emph{frozen}. 
Observe that a put operation can successfully complete in chunk $C$ at a time $t$ only if  (1) $C$ is accessible  at some time $t'<t$, 
and (2) $C$ is not frozen at time $t$. This is because once a thread's PPA entry is frozen, its attempt to CAS it inevitably fails and it triggers rebalance.
We say that a chunk is \emph{mutable} if these two conditions are satisfied. Similarly, a chunk is \emph{immutable} before it first becomes accessible 
and again after the freezing stage of its rebalance is complete. 

Rebalance preserves the following invariant:
\begin{invariant}
At any point in an execution of \kiwi, for every key $k$,  
\begin{enumerate}
\item the \code{minKey} values in the linked list are monotonically increasing (so $k$ is in the range of exactly one accessible chunk);  
\item  $k$ is in the range of at most one mutable chunk; and 
\item querying the index for $k$  returns a chunk $C$ s.t.\  $C$.\code{minKey} $\le k$ and $C$ is either accessible or frozen.
\end{enumerate}
\label{invariant:rebalance}
\end{invariant}
\begin{proof}
\begin{enumerate}
\item
Observe that when a segment of new chunks is connected instead of a sequence of old ones, $C_f$.\code{minKey} is equal to $C$.\code{minKey}, 
and  the \code{next} pointer in $C$'s predecessor is replaced via CAS from $C$ to $C_f$ (line~\ref{l:set-pred}) hence the invariant is preserved on the left side of the new segment.
The invariant is also preserved on the right side of the new segment because each new chunk's \code{minKey} is set to some key encountered in the old segment before \code{last}, 
and $C_n$.\code{next} is guaranteed to be \code{last.next} (lines  \ref{l:mark}--\ref{l:mark-end}).  
\item
The rebalance protocol does not link new chunks to the list (stage (5)) 
before freezing the old chunks holding the same key range (stage (2)).
Moreover, once a chunk is engaged (stage (1)), it is associated with a unique rebalance object \code{ro}
whose next pointer is set to $\bot$, 
and hence the segment of chunks associated with \code{ro} cannot change. 
Using a CAS to set $C$'s predecessor \code{next} pointer to $C_f$ ensures 
that the old immutable chunk is replaced by at most one new accessible mutable chunk.
\item
Chunks are indexed according to their \code{minKey}, and 
the rebalance protocol does not index new chunks (stage (6)) before making them accessible (stage (5)). 
Before a chunk ceases to be accessible, it must be frozen. 
\end{enumerate}
\end{proof}

In addition to preserving the data structure's integrity, rebalance ensures that no key-value pairs disappear from the data structure due to rebalancing. 
We say that a key-value pair $\langle$key, val$ \rangle$ is \emph{stored in} \kiwi\ at time $t$ in execution $\sigma$ if invoking get(key) at the end of $\sigma$
and allowing it to complete without interfering steps of other threads returns val. We show the following:

\begin{proposition}
If $\langle$key, val$ \rangle$ is \emph{stored in} \kiwi\ at time $t$ in execution $\sigma$ and no subsequent put(key,$\_$)  operations are invoked in $\sigma$, 
then $\langle$key, val$ \rangle$ is \emph{stored in} \kiwi\ at all times $t' > t$ in  $\sigma$.
\label{proposition:no-loss}
\end{proposition}
\begin{proof}
By Invariant~\ref{invariant:rebalance}(1), key is in the range of exactly one accessible chunk $C$ at time $t$,  which  get(key) locates, 
and the returned val is the one associated with key with the highest version (with ties broken by valPtr). 
Observe that as long as $C$ remains accessible at time $t'$, its range does not change because \code{minKey} is invariant, and if $C$'s successor is 
replaced by rebalance, it is replaced with a chunk with the same \code{minKey}. 

Since no   subsequent put(key,$\_$)  operations are invoked in $\sigma$, val remains the highest-version value associated with key in $C$, and we are done.
It remains to show that a rebalance that removes $C$ does not remove $\langle$key, val$ \rangle$ from \kiwi, from which the proposition follows   inductively.
%Consider therefore a rebalance that engages $C$.  %Once $C$.\code{ro} is set, it does not change.
This, in turn, follows from the facts that (1) the highest-versioned value associated with each key in an old chunk $C$ is cloned into a new chunk; and 
(2) the entire chunk segment is replaced atomically by first marking the next pointer of the last engaged chunk to prevent it from changing, and then 
CASing the predecessor of the first engaged chunk.
 \end{proof}

\subsubsection{Puts.} 
\label{ssec:put-proof}

Puts in a chunk $C$
are ordered (lexicographically) according to their version-value-index   pairs $\langle v, j \rangle$, where
$\langle v, i \rangle$ is published in the appropriate  PPA entry in phase 2 of the put, and \code{$C$.k[$i$].valPtr$=j$}; this pair is called the \emph{full version} of the put.
We note that in each chunk, the full versions are unique, because threads obtain $j$ using F\&A (line~\ref{l:put-cas-j}).
First,  $i$ is published in \code{ppa[t].idx} (line~\ref{l:put-version}) and then
the  pair gets its final value by a successful CAS of \code{ppa[t].ver}, either by the put (line~\ref{l:put-cas-version}) or by a helping thread (line~\ref {l:get-help}). 
We refer to 
a step publishing $i$ in \code{ppa[t].idx} and to
the step executing the successful CAS  as the put's \emph{publish time} and the \emph{full version assignment time}, resp., 
and say that the put \emph{assigns} full version $\langle v, j \rangle$ for its key in $C$.

We note that each put assigns a full version at most once. 
Furthermore, as noted above, a full version can only be assigned in a mutable chunk.
Once a put operation $po$ for key $k$ assigns its full version in chunk $C$ at time $t$, we can define its linearization point. 
There are two options: 
\begin{enumerate}
\item If at time $t$
$po$'s full version $\langle v, j \rangle$ is the highest for $k$ in $C$, (among entries in $C$'s {PPA} and  linked list),
then  \lp{po} is the last step reading $v$ from \code{GV} before $t$ (line~\ref{l:put-LP} or~\ref{l:put-helped-LP}). 
\item  Otherwise, let $po'$ be the 
\code{put($k,\_$)} operation that assigns for $k$ in $C$ the smallest full version exceeding $po$'s
before time $t$. Then \lp{po} is recursively defined to be \lp{po'}. Note that 
$po$'s full version assignment time exceeds that of  $po'$, so the recursive definition does not induce cycles. 
In case multiple puts are assigned to the same point, they are linearized in increasing full version order.
\end{enumerate}



%% HERE %%

% It is easy to show that a put operation always lands at a mutable chunk with a range that covers the key.
By Invariant~\ref{invariant:rebalance},
rebalance operations divide puts of key $k$ into disjoint groups; one group per mutable epoch of each chunk covering the key.
The following lemma 
establishes the order among linearization points of puts within one epoch.
\begin{lemma}
\label{proof:put}
Consider chunk $C$ accessible as of time $t_0$, key $k$ in the range of $C$, and an
operation $po=$\code{put($k, \_$)} that assigns $\langle v, j\rangle$ to $C$.\code{ppa} at time $t$. Then  
\begin{enumerate}
\setlength{\itemsep}{0pt}
\setlength{\parskip}{0pt}
\item \label{proof:put:lp1} \lp{po} is after $po$ allocates location $j$ for its value and before $t$.
\item \label{proof:put:lp2} \lp{po} is a read step of \code{GV} that returns $v$.
\item \label{proof:put:lp3} \lp{po} is after some operation $po'$ (possibly $po$, but not necessarily) publishes for $k$ to $C$ where later $po'$ assigns a full version equal to or greater than $\langle v, j\rangle$.
\item \label{proof:put:lp4} The linearization points of all operations that publish for $k$ to $C$ preserve their full version order.
\item \label{proof:put:lp5} At time $t_0$, the value published to $k$ by the put with the latest linearization point before $t_0$ is associated with the highest full version in C's linked list.
\end{enumerate}
\end{lemma}


%For every key k and time t when k is in the range of a chunk C that is accessible (either from the index or from the chunks list), the value published to k by the put with the latest LP before t is associated with the highest location-based version in C's linked list and PPA.


\subsubsection{Gets and scans.}
\label{ssec:get-proof}

The most subtle linearization is of get operations.
A get operation $go$ may land in a mutable or immutable chunk. 
We need to linearize $go$ before all concurrent puts that $go$ misses while seeking the value.
%get
For a get operation $go$ for a key $k$ in the range of chunk $C$, there are three options:
\begin{enumerate}
\setlength{\itemsep}{0pt}
\setlength{\parskip}{0pt}
\item If $C$ is not accessible from the chunks list when $go$ starts traversing $C$'s PPA, then \lp{go} is the last step in which $C$ is still accessible from the chunks list.
\item Else, if $go$ does not find $k$ in $C$ then \lp{go} is when $go$ starts traversing $C$'s PPA.
\item Else, let $po$ be the put operation that inserts the value returned by $go$. \lp{go} is the latest between when $go$ starts traversing $C$'s {PPA} and immediately after \lp{po}.
\end{enumerate}

The next lemma shows that in the third case no other put writing to $k$ is linearized after \lp{po} and before \lp{go}.
The proof relies on %Conditions~\ref{proof:put:lp1},~\ref{proof:put:lp3} and~\ref{proof:put:lp4} of 
Lemma~\ref{proof:put} and the rebalance invariants.% to prove it.

\begin{lemma}
\label{proof:get}
Consider a get operation $go$ retrieving the value of key $k$ from chunk $C$. Let $t$ be the step in which $go$ starts traversing $C$'s {PPA}. Then:
\begin{enumerate}
\setlength{\itemsep}{0pt}
\setlength{\parskip}{0pt}
\item \label{proof:get:lp1} If $go$ does not find $k$ in $C$, then for each operation $po$ publishing $k$ in $C$, \lp{po} is after $t$.
\item \label{proof:get:lp2} If $go$ returns the value written by operation $po$, then \lp{go} is after \lp{po}, and for each  $po' \neq po$ publishing $k$ in $C$, \lp{po'} is either before \lp{po} or after $t$.
\end{enumerate}
\end{lemma}

Scans are linearized when \code{GV} is increased beyond their read point, typically by their own F\&I, and sometimes by a helping rebalance. 
%We use %Conditions~\ref{proof:put:lp2} and~\ref{proof:put:lp4} of 
Lemma~\ref{proof:put} helps to prove the following:
\begin{lemma}
\label{proof:scan}
Consider a scan operation $so$ that acquires version $v$ as its read point. For each key $k$ in the range of the scan, $so$ returns the value of the put operation writing to $k$ that is linearized last before \lp{so}.
\end{lemma}

The definition of the linearization points of scans and get operations imply that these operations are linearized between their invocation and return.
Condition~\ref{proof:put:lp1} of Lemma~\ref{proof:put} implies the same for puts. 
It is easy to show that gets and scans land in chunks that contain the saught keys in their ranges. Combined with the rebalancing invariants,
Lemma~\ref{proof:get} shows that get operations satisfy their sequential specification, and Lemma~\ref{proof:scan} proves that scans satisfy their sequential specification. 
Hence we conclude that \kiwi\ implements a linearziable map. 


\subsection{Liveness}
\label{sec:live}


The proof shows that (1) every get and scan completes within a finite number of steps, and (2) in every execution, \emph{some} put operation completes. We omit it for lack of space.

 {\kiwi}'s gets and scans are \emph{wait-free}, namely, in any execution, every operation completes within a finite number of steps by its invoking thread. The proof shows that the number of iterations in the loops in these operations is finite. 

We further prove that put operations are \emph{lock-free}, namely, in every execution, \emph{some} operation completes. We show that although a put operation can execute an infinite number of rebalances, this occurs because 
some operation (and in fact many operations) successfully complete a put.




\subsection{Liveness}
\label{sec:live}

We now prove \kiwi's liveness properties. 
We show that gets and scans are \emph{wait-free}, namely, in any execution, every operation completes within a finite number of steps by its invoking thread. 
 This is proven  by showing that the number of iterations in the loops in these procedures is finite. 
 Put operations satisfy a weaker liveness property --  \emph{lock-freedom}, namely, in every execution, \emph{some} put operation completes.
To prove this, we show that although a put operation can execute an infinite number of rebalances, this can only occur because 
some other operation (and in fact many operations) successfully complete a put.

We begin by showing that the \codeF{locate} function is wait-free. 
\begin{proposition}  
The  \code{locate} function is wait-free.
\label{prop:locate}
\end{proposition}
\begin{proof}
The function consists of two loops. The first loop 
executes a finite number of iterations because \code{index.lookup(key)} always returns a chunk $C_0$ with $C_0$.\code{minKey} $\le$ key,
and so the searched keys are monotonically decreasing.  Since the number of keys is finite and the head of the list is never frozen, the loop
returns an unfrozen chunk after a finite number of iterations. The ensuing traversal is also finite because the linked list is finite. 
\end{proof}

\begin{lemma}
The  \code{get} and \code{scan} functions are wait-free.
\end{lemma}
\begin{proof}
By Proposition~\ref{prop:locate}, the \code{locate} function is wait-free. 
To complete the proof, observe that the loops performed in \code{helpPendingPuts} and \code{findLatest}
 iterate (or search) over finites sets -- $C.k$ and $C.ppa$, and the scan loop (line~\ref{l:scan-loop}) traverses
 chunks in the (finite) linked list. 
\end{proof}

\begin{lemma}
The  \code{put} function is lock-free.
\end{lemma}
\begin{proof}
We first show that as long as no \code{rebalance} occurs, \code{put} completes within a finite number of steps. 
Beyond \code{rebalance} calls and recursive calls to \code{put} when it fails, \code{put} executes 
two loops --  one in the \code{locate} function and one 
%only one loop, 
in phase 3 (lines~\ref{l:put-ll}--\ref{l:put-ll-end}).  
The former is wait-free by  Proposition~\ref{prop:locate}. 
Next, consider the loop in phase 3 of the \code{put}. 
Observe that the \codeF{find} function is wait-free because it searches a finite array ($C$.k).
Moreover, the CAS in line~\ref{l:put-insert1} fails at most a finite number of times (again, due to the finality of  $C$.k),
and because we break when it succeeds, it is attempted a finite number of times. 
Additionally, every time the CAS in line~\ref{l:put-insert2} is attempted (and either succeeds or fails),   \codeF{c.valPtr} increases,
implying that after a finite number of attempts, {\codeF{c.valPtr} $ \geq$ \codeF{j}}, and the loops completes. 

Next, consider \code{rebalance} triggered by the \code{put}. This can occur in \code{checkRebalance} (called in line~\ref{l:put-rebalance0}), 
or when either $C.k$ or $C.v$ is full (line~\ref{l:put-full}), or when the chunk is already frozen (line~\ref{l:put-restart}). 
To prove lock-freedom, 
we will show that every time \code{rebalance} returns false (leading to a recursive call to \code{put}) or iterates  
an additional time in an unbounded loop, or recursively calls \code{rebalance}, this is because some new  \code{put} returned successfully. 
Note in particular, that if a  \code{rebalance} operation succeeds to replace the engaged sequence (and returns true), then it 
completes the \code{put} that invoked it, and so we do not need to consider this case.

We first argue that with the exception of rebalance stage 5 (replace), all other  \code{rebalance} stages  always complete within a finite number of steps. 
Because the chunks list is finite, the two loops in stage 1 (engage) are finite. 
Similarly, the list of engaged chunks and the number of PPA entries in each of them are finite, and so is the nested loop in stage 2 (freeze). 
In stage 3 (pick minimal version), the loop over the PSA is finite, and it constructs a finite toHelp set, and so the loop over toHelp is finite as well.
Likewise, the loops in stage 4 iterate over finite sets, namely, engaged chunks and keys in the data structures therein.  
In stage 6, the conditional insertion to the index fails only if another thread succeeds to insert a chunk with the same \code{minKey}. 
Because the insertion is retried only as long as the chunk is not frozen, interfering insertions to the index must be due older rebalances that began before 
the current rebalance, of which there is  a finite number.
Clearly, stage 7, which performs a single CAS, is also wait-free. 
Hence, it remains to consider stage 5.   

The first loop in stage 5 terminates once the CAS -- either of the thread executing the loop or of another thread -- succeeds to mark \code{last.next} as immutable. 
The CAS fails if either \code{last.next} changes or another thread marks it.  
But  \code{last.next} only changes if some \code{rebalance} succceds to replace an engaged chunk succeeding it, and this  \code{rebalance} 
returns true upon completion, thus completing a \code{put} when it returns. 

The second loop returns whenever either the current thread or another thread successfully replaces the \code{next} pointer at C's predecessor to 
a node whose parent is $C$ ($C_f$ in case it is the current thread). 
They fail to do so only in case the predecessor's  \code{next} pointer is marked, which in turn only occurs if the predecessor itself is engaged in 
another rebalance. In this case, the current thread makes a recursive call to  \code{rebalance} in order to help the predecessor (line~\ref{l:nested-rebalance}), 
and then executes another iteration of the loop. 
Note that since the linked list is finite, the depth of the recursion is finite, so eventually the inner-most nested recursive call succesfully returns. 
Once a nested call in line~\ref{l:nested-rebalance} returns after having replaced \code{pred} with a new chunk \code{pred'}, 
either some thread will succeed to CAS the new predecessor's next pointer causing 
the calling thread to exit the loop in the next iteration, or  the new chunk \code{pred'}  will undergo rebalance.
However, for the latter to occur, it must be the case that at least one new \code{put}  has completed in  \code{pred'} following its rebalance.
This is because when rebalance completes, \code{pred'.k} is sorted and  \code{pred'.k} and \code{pred'.v} are at most half full, and 
because they hold more entries than twice the number of threads,   it is impossible for either of them to fill up due to pending put operations. 
Thus, \code{rebalance} can only be required after a successful   \code{put} in   \code{pred'}.

\remove{

We prove by contradiction.
Assume that \emph{put} is not lock-free.
Hence there exists an infinite execution $\pi = C_0,s_1,C_1,\ldots$ such that
after configuration $C_{k0}$ ($k0 \geq 0$) no operation completes and no new operation is invoked.
%no operation completes after configuration $C_{k0}$.
%
We have already shown that \emph{get} and \emph{scan} are wait-free,
 therefore there exists configuration $C_{k1}$ in $\pi$ ($k0 \leq k1$) such that after $C_{k1}$ all the running threads execute  \emph{put} operations.


We write \emph{$t:X$.allocate} to denote an invocation of \emph{allocate} on chunk $X$ by thread $t$
(\emph{allocate} is invoked by function \emph{put} at line $6$).
We say that invocation \emph{t:$X$.allocate} is \emph{successful}
if it returns a reference to a valid cell (i.e., it returns a non-null value).

Consider an execution of the loop in function \emph{put} by thread $t$.
If \emph{$t:X$.allocate} is not successful at iteration $i+1$ ($i \geq 1$) of $t$ ,
then at iteration $i$ the invocation of \emph{$t:Y$.allocate} is also not successful.
Hence thread $t$ invokes \emph{Rebalance($Y$)} at the end of iteration $i$.
Therefore $Y \neq X$ and chunk $X$ has been added to kiwi (at some point) during iterations $i$ and $i+1$ of $t$
(when a chunk is added to kiwi, this chunk is not full).
Therefore another thread $t' \neq t$ successfully invoked \emph{$t':X$.allocate} (at some point) during iterations $i$ and $i+1$ of $t$
(otherwise \emph{$t:X$.allocate} should be successful).
%
Since a thread do not start new iteration of this loop after a successful invocation of \emph{allocate} (see lines $6$--$9$ in \emph{put}),
 we know that there exists configuration $C_{k2}$ in $\pi$ ($k1 \leq k2$) such that:
after $C_{k2}$ the \emph{put} operations do not start new iterations of this loop.

Since \emph{allocate} is wait-free, there exists configuration $C_{k3}$ in $\pi$ ($k2 \leq k3$) such that
no thread executes \emph{allocate} after $C_{k3}$.
Hence, no chunk becomes full in $\pi$ after configuration $C_{k3}$.


%Consider the loop in function \emph{put}.
%This loop may have iteration $i+1$ only if \emph{chunk.allocate} is not successful at iteration $i$ ---
%this can only happen if (at some point) after the beginning the $i$-th iteration a concurrent \emph{put} operation has successfully invoked \emph{chunk.allocate} on the same chunk.
%Hence there exists configuration $C_{k2}$ in $\pi$ ($k1 \leq k2$) such that: after $C_{k2}$ the \emph{put} operations do not start new iterations of this loop.

In the following paragraphs we focus on the functions \emph{Kiwi::Rebalance} and \emph{Kiwi::Replace}.
Notice that \emph{Kiwi::Rebalance} calls to \emph{Kiwi::Replace}; and that \emph{Kiwi::Replace} recursively calls to \emph{Kiwi::Rebalance} in line $20$.
An invocation of \emph{Kiwi::Rebalance} on chunk $X$ removes $X$ from kiwi:
hence for each thread $t$ and chunk $X$, \emph{Rebalance($X$)} may be invoked at most once by $t$.

Let $N_{puts}$ be the number of \emph{put} operations after configuration $C_{k3}$.
Let $N_E$ be the maximal number of new chunks created by an invocation of \emph{balance()} (in our implementation $N_E=4$).
Since after $C_{k3}$ no chunk may become full, at most $N_{puts} \times N_E$ new chunks may be created after $C_{k3}$.
Therefore, after configuration $C_{k3}$ the threads invoke function  \emph{Kiwi::Rebalance} a finite number of times.
Hence there exists a configuration $C_{k4}$ in $\pi$ ($k3 \leq k4$) such that \emph{Kiwi::Rebalance} is not invoked after $C_{k4}$.



%Let $N_{puts}$ be the number of \emph{put} operations after configuration $C_{k2}$.
%After $C_{k2}$, at most $N_{puts}$ chunks may become full.
%If $N_E$ is the maximal number of new chunks created by an invocation of \emph{balance()},
%then after configuration $C_{k2}$ at most $N_{puts} \times N_E$ new chunks may be created.
%Therefore, after configuration $C_{k2}$ the threads invoke function  \emph{Kiwi::Rebalance} a finite number of times.
%Hence there exists a configuration $C_{k3}$ in $\pi$ ($k2 \leq k3$) such that \emph{Kiwi::Rebalance} is not invoked after $C_{k3}$.

%Consider the loop in \emph{Kiwi::Replace}.

Since the other functions invoked by \emph{put} are lock-free\footnote{
The other functions invoked by \emph{put}, \emph{Kiwi::Rebalance} and \emph{Kiwi::Replace} are lock-free:
we assume that \emph{index.replace} is lock-free, the other ones are trivially wait-free.},
there exists a configuration $C_{k5}$ in $\pi$ ($k4 \leq k5$) such that all the running threads
after $C_{k5}$ execute the loop in \emph{Kiwi::Replace} (and this loop is never terminated).
%
Hence, the CAS at line $22$ always fails after configuration $C_{k5}$ (i.e., no CAS updates shared memory after $C_{k5}$).
This is a contradiction, because the  CAS at line $22$ may fail only if shared memory has been updated since the beginning of the iteration.



%
%
%
%Consider the loop in \emph{Kiwi::Replace}.
%Since the other functions are lock-free, there exists a configuration $C_{k4}$ in $\pi$ ($k3 \leq k4$) such that all the threads which are running after $C_{k4}$ do not exit from this loop.
%Hence, the CAS at line $22$ always fails after configuration $C_{k4}$ (i.e., no CAS updates shared memory after $C_{k4}$).
%This is a contradiction, because the  CAS at line $22$ may fail only if shared memory has been updated since the beginning of the iteration.
}
\end{proof}

\section{Evaluation}
\label{sec:eval}

\subsection{Setup}

\begin{figure*}
\begin{center}
\begin{tikzpicture}
\begin{axis}[
        mystyle,
        unbalanced,
        title={(a) Get  },
        ylabel = { Throughput, M keys/sec}, 
        xlabel = { Threads\\ }
]
\addplot [black,mark=square*] table [x={threads}, y={KiWi}]
{results/get.txt}; 
\addplot [orange,mark=diamond] table [x={threads}, y={KAry}]
{results/get.txt}; 
\addplot [blues5,mark=10-pointed star] table [x={threads}, y={JavaSkipList}]
{results/get.txt}; 
\addplot [magenta,mark=+] table [x={threads}, y={SnapTree}]
{results/get.txt};
\end{axis}
\end{tikzpicture}
\begin{tikzpicture}
\begin{axis}[
        mystyle,
        unbalanced,
        title={ (b) Put  },
        xlabel = { Threads\\ }
]
\addplot [black,mark=square*] table [x={threads}, y={KiWi}]
{results/insert-delete.txt}; 
\addplot [orange,mark=diamond] table [x={threads}, y={KAry}]
{results/insert-delete.txt}; 
\addplot [blues5,mark=10-pointed star] table [x={threads}, y={JavaSkipList}]
{results/insert-delete.txt}; 
\addplot [magenta,mark=+] table [x={threads}, y={SnapTree}]
{results/insert-delete.txt};
\end{axis}
\end{tikzpicture}
%%%%%%%%%%%%%%%%%%%
%%%%%%%%%%%%%%%%
\begin{tikzpicture}
\begin{axis}[
	mystyle,
	scansOnly,
	title= {\small{ (c) Scan (32K keys) }},
        xlabel = { Threads\\ }
]
\addplot [black,mark=square*] table [x={threads}, y={KiWi}]
{results/scans32K.txt};
\addplot [orange,mark=diamond] table [x={threads}, y={KAry}]
{results/scans32K.txt};  
\addplot [blues5,mark=10-pointed star] table [x={threads}, y={JavaSkipList}]
{results/scans32K.txt}; 
\addplot [magenta,mark=+] table [x={threads}, y={SnapTree}]
{results/scans32K.txt};
\end{axis}
\end{tikzpicture}
\ref{scansOnlyLegend}
\end{center}
\caption{Throughput scalability with uniform workloads. (a) Get operations, (b) Put operations, and (c) Scan operations. }
\label{evaluation:results:getputscan}
\end{figure*}

\begin{figure*}
\begin{center}
\begin{tikzpicture}
\begin{axis}[
	mystyle16,
	unbalanced,
	title= {\small{(a) Scan 32K keys (parallel put), 1M}},
	ylabel = { Scan throughput, M keys/sec},
	xlabel= {Scan threads\\}
]
\addplot [black,mark=square*] table [x={threads}, y={KiWi}]
{results/put_scan32K.txt};
\addplot [orange,mark=diamond] table [x={threads}, y={KAry}]
{results/put_scan32K.txt};  
\addplot [blues5,mark=10-pointed star] table [x={threads}, y={JavaSkipList}]
{results/put_scan32K.txt}; 
\addplot [magenta,mark=+] table [x={threads}, y={SnapTree}]
{results/put_scan32K.txt};
\end{axis}
\end{tikzpicture}
%\hspace{.05\textwidth}
\begin{tikzpicture}
\begin{axis}[
	rangestyle_ext,
	title={\small{ (b) Scan 16 threads (parallel put), 1M}},
	xlabel= {Scan range (keys)\\}
]
\addplot [black,mark=square*] table [x={ranges}, y={KiWi}]
{results/scans_mixed_by_range.txt}; 
\addplot [orange,mark=diamond] table [x={ranges}, y={KAry}]
{results/scans_mixed_by_range.txt}; 
\addplot [blues5,mark=10-pointed star] table [x={ranges}, y={JavaSkipList}]
{results/scans_mixed_by_range.txt}; 
\addplot [magenta,mark=+] table [x={ranges}, y={SnapTree}]
{results/scans_mixed_by_range.txt};
\end{axis}
\end{tikzpicture}
\begin{tikzpicture}
\begin{axis}[
	rangestyle_ext,
	title={\small{ (c) Scan 16 threads (parallel put), 10M}},
	xlabel= {Scan range (keys)\\}
]
\addplot [black,mark=square*] table [x={ranges}, y={KiWi}]
{results/scans_bg_puts_by_range_10M.txt}; 
\addplot [orange,mark=diamond] table [x={ranges}, y={KAry}]
{results/scans_bg_puts_by_range_10M.txt}; 
\addplot [blues5,mark=10-pointed star] table [x={ranges}, y={JavaSkipList}]
{results/scans_bg_puts_by_range_10M.txt}; 
\addplot [magenta,mark=+] table [x={ranges}, y={SnapTree}]
{results/scans_bg_puts_by_range_10M.txt};
\end{axis}
\end{tikzpicture}

\begin{tikzpicture}
\begin{axis}[
	mystyle16,
	unbalanced,
	xlabel= { Put threads\\},
	ylabel = { Put throughput, M keys/sec },
	title={\small{(d) Put (parallel scan 32K keys), 1M}}
]
\addplot [black,mark=square*] table [x={threads}, y={KiWi}]
{results/s_put32K.txt};
\addplot [orange,mark=diamond] table [x={threads}, y={KAry}]
{results/s_put32K.txt};  
\addplot [blues5,mark=10-pointed star] table [x={threads}, y={JavaSkipList}]
{results/s_put32K.txt}; 
\addplot [magenta,mark=+] table [x={threads}, y={SnapTree}]
{results/s_put32K.txt};
\end{axis}
\end{tikzpicture}
%\hspace{.05\textwidth}
\begin{tikzpicture}
\begin{axis}[
	rangestyle_ext,
	xlabel= {Scan range (keys)\\},
	title= {\small{(e) Put 16 threads (parallel scan), 1M}}
]
\addplot [black,mark=square*] table [x={ranges}, y={KiWi}]
{results/puts_mixed_by_range.txt}; 
\addplot [orange,mark=diamond] table [x={ranges}, y={KAry}]
{results/puts_mixed_by_range.txt}; 
\addplot [blues5,mark=10-pointed star] table [x={ranges}, y={JavaSkipList}]
{results/puts_mixed_by_range.txt}; 
\addplot [magenta,mark=+] table [x={ranges}, y={SnapTree}]
{results/puts_mixed_by_range.txt};
\end{axis}
\end{tikzpicture}
\begin{tikzpicture}
\begin{axis}[
	rangestyle_ext,
	xlabel= {Scan range (keys)\\},
	title= {\small{(f) Put 16 threads (parallel scan), 10M}}
]
\addplot [black,mark=square*] table [x={ranges}, y={KiWi}]
{results/puts_bg_scans_by_range_10M.txt}; 
\addplot [orange,mark=diamond] table [x={ranges}, y={KAry}]
{results/puts_bg_scans_by_range_10M.txt}; 
\addplot [blues5,mark=10-pointed star] table [x={ranges}, y={JavaSkipList}]
{results/puts_bg_scans_by_range_10M.txt}; 
\addplot [magenta,mark=+] table [x={ranges}, y={SnapTree}]
{results/puts_bg_scans_by_range_10M.txt};
\end{axis}
\end{tikzpicture}

\ref{scansOnlyLegend}

\end{center}
\caption{Throughput scalability with concurrent scans and puts. (a,b) Scan operations. (c,d) Put operations. }
\label{evaluation:results:scan}
\end{figure*}

{\bf Implementation.} We implement {\kiwi} in Java, using Doug Lea's concurrent skip-list 
implementation~\cite{JavaConcurrentSkipList} for the index with added locks to support conditional updates. 
The code makes extensive use of efficient 
array copy methods~\cite{JavaArrayCopy}. {\kiwi}'s chunk size is set to 1024. 

The rebalance policy is tuned as follows:
$\code{checkRebalance}$  invokes rebalance with probability $0.15$ whenever
 the batched prefix consists of less than $0.625$ of the linked list. Rebalance 
engages the next chunk whenever engaging it will reduce the number of chunks in the list. 

We did not implement the piggybacking of puts on rebalance, and instead restart the put after every rebalance.
This does not violate lock-freedom because the number of threads is much smaller than the chunk size, and 
hence it is impossible for pending put operations to fill up an entire chunk.

\textbf{Methodology.}
We leverage the popular {\em synchrobench}  microbenchmark~\cite{Gramoli2015}
to exercise a variety of workloads. The hardware platform %is a high-performance server, 
features four Intel Xeon E5-4650 8-core CPUs. %(i.e., 32 hardware threads overall). %hyperthreading disabled). 
Every experiment starts with 20 seconds of {\em warmup} -- inserts
and deletes of random keys -- to let the HotSpot compiler optimizations take effect. 
It then runs 10  iterations, 5 seconds each, and averages the results. An iteration fills the map with 1M random (integer, integer) pairs, 
then exercises some workload. %scenario characterized by the selection of API's, their parameters, and degree of parallelism. 

\textbf{Competition.}
We compare {\kiwi} to Java implementations of three concurrent KV-maps: (1) the traditional 
skip-list~\cite{JavaConcurrentSkipList} which does not provide linearizable scan semantics, 
(2) {\kary}~\cite{BrownA12}\footnote{\small{\url{http://www.cs.toronto.edu/~tabrown/kstrq/LockFreeKSTRQ.java}}.}, 
and (3) {\snaptree}\cite{BronsonCCO2010}\footnote{\small{\url{https://github.com/nbronson/snaptree}}.}. 
%The implementations of {\kary} and {\snaptree} are available online. 
For {\kary}, we use the optimal configuration described in~\cite{BrownA12} with $k=64$. 

\subsection{Results}


\begin{figure}
\begin{center}
\begin{tikzpicture}
\begin{axis}[
	memstyle16,	
	title= {\small{Scan (32 keys) with background put }},
	xlabel = {Scan threads\\},
	ylabel = {Memory footprint (MB)}
]
\addplot [black,mark=square*] table [x={threads}, y={KiWi}]
{results/mem_p_s_r32.txt};
\addplot [orange,mark=diamond] table [x={threads}, y={KAry}]
{results/mem_p_s_r32.txt};  
\addplot [blues5,mark=10-pointed star] table [x={threads}, y={JavaSkipList}]
{results/mem_p_s_r32.txt}; 
\addplot [magenta,mark=+] table [x={threads}, y={SnapTree}]
{results/mem_p_s_r32.txt};
\end{axis}
\end{tikzpicture}

\ref{singleKeyOpsLegend}
\end{center}
\caption{RAM footprint with concurrent scans and puts. }
\label{evaluation:results:mem}
\end{figure}

\textbf{Basic scenarios: get, put, and scan.} 
We first focus on three simple workloads: 
(1) get-only (random reads), 
(2) put-only (random writes, half inserts/updates and half deletes), and 
(3) scan-only (sequential reads of 32K keys, each starting from a random lower bound).  
%The keys that define the operations are selected uniformly at random. 

Figure~\ref{evaluation:results:getputscan} depicts throughput scalability with the number of worker threads. 
In get-only scenarios (Figure~\ref{evaluation:results:getputscan}(a)), {\kiwi} outperforms the other 
algorithms by 1.25x to 2.5x. We explain this by the NUMA- and cache-friendly locality of access in its intra-chunk binary search. 
Under put-only workloads (Figure~\ref{evaluation:results:getputscan}(b)), it also performs well, thanks to avoiding version
manipulation. {\snaptree}, which is optimized for random writes, is approximately $10\%$ faster than {\kiwi}
with 32 threads. Note that in general, {\kiwi}'s gets are faster than its puts because the latter occasionally incur rebalancing. 

Finally, {\kiwi} excels in scan performance (Figure~\ref{evaluation:results:getputscan}(c)). 
For example, with 32 threads, it exceeds its closest competitor, {\kary}, by over $40\%$. 
Here too, {\kiwi}'s advantage stems from high locality of access while scanning big chunks. 

\textbf{Concurrent scans and puts.}
We now turn to the scenario that combines analytics (scan operations) with 
real-time updates (put operations). This is the primary use case that motivated 
the design principles behind {\kiwi}. Half of the threads perform scans, whereas 
the second half performs puts. 

Figure~\ref{evaluation:results:scan}(a) depicts scan throughput scalability with the number of threads
while scanning ranges of 32K keys. Figure~\ref{evaluation:results:scan}(b) depicts the throughput for 16 scan 
threads with varying range sizes. Note that for long scans, {\kary}'s performance deteriorates under contention. 
This happens because {\kary} restarts the scan every time a put conflicts with it -- i.e., puts make progress 
but scans get starved. For large ranges, {\snaptree} has the second-fastest scans because it shared-locks 
the scanned ranges in advance and iterates unobstructed. However, this comes at the expense of puts, 
since such locking starves concurrent updates. Figures~\ref{evaluation:results:scan}(c) 
and~\ref{evaluation:results:scan}(d) illustrate this phenomenon. 

We study the memory footprints of the solutions in this scenario. We focus on 32-key scans -- a setting 
in which the throughput achieved by all the algorithms except {\snaptree} is similar.  
Figure~\ref{evaluation:results:mem}(e) depicts the JVM memory-in-use metric immediately after a full 
garbage collection that cleans up all the unused objects, averaged across 50 data points. 
{\kiwi} is on par with {\kary} and the Java skiplist except with maximal parallelism (16 put threads), 
in which it consumes 20\% more RAM due to intensive version management.  

%Summing up, contrast to its competitors that are tailored for certain workloads, 
%{\kiwi} serves both scans and puts equally well, echoing the theoretical results.  

\textbf{Ordered workload.} 
As a balanced data structure, {\kiwi} provides good performance on non-random workloads. 
We experiment with a monotonically ordered stream of keys. {\kiwi} achieves a throughput 
similar to the previous experiments. In contrast, {\kary}'s maximal put throughput in this setting is 
730 times slower -- approximately $13.6$K operations/sec vs {\kiwi}'s $9.98$M. 

\section{Discussion}
\label{sec:disc}

We presented {\kiwi}, a KV-map tailored for real-time analytics applications. \kiwi\ is the first
concurrent KV-map to support high-performance atomic scans simultaneously with real-time updates of the data.
In contrast to traditional approaches, {\kiwi} shifts the synchronization
overhead from puts to scans, and offers lock-free puts and wait-free gets and scans.
We demonstrated {\kiwi}'s significant performance gains over state-of-the-art KV-map
implementations that support atomic scans. 

\inote{
Future work could apply {\kiwi} as the in-memory part of a NoSQL KV-store, e.g., HBase~\cite{ApacheHBase}.
Prior results~\cite{GolanGueta2015} highlighted that the overall  performance of such systems can be boosted by
improving  concurrency at the memory part.%in-memory KV-store.
}

\bibliographystyle{abbrv}
\bibliography{ref}
%\bibliographystyle{abbrv}
%\bibliographystyle{abbrvnat}
%\appendix
%\section{Proof Appendix}
\label{app:proof}

\subsection{Safety}
\label{app:proof:safety}

%\par{Lemma~\ref{proof:put} Proof.}
\begin{proof} \textbf{of Lemma~\ref{proof:rebalance}.}
\end{proof}

\begin{proof} \textbf{of Lemma~\ref{proof:put}.}
We consider an execution interval $\pi$ which spans the execution intervals of all operations in $OP_k^c$.
Denote by $\sigma_1, \sigma_2, \ldots $ the finite sequence of steps of these operations writing versions $v_1, v_2, \ldots$ into entries in the \code{ppa} by their order in $\pi$, where $\sigma_i$ is a step of operation $op_i$; the locations each operation allocated for its value are $j_1, j_2, \ldots$, respectively.

The proof is by induction on $i$. For the base case, we consider $op_1$. It is the first to publish its version in the \code{ppa}. Lemma~\ref{lemma:rebalance:infant} implies that all the cells within the chunk range that were inserted into an earlier chunk were added to the chunk's cell linked list by a rebalance operation that completed before $op_1$ started. Therefore, these cells have smaller location-based versions and $op_1$ is linearized in the last read step of the global version counter that returns $v_1$ before $\sigma_1$. Clearly this step is after the put operation is published, and specifically after $j_1$ is allocated, and the lemma holds.

For the induction step, assume the lemma holds for operations $op_1, \ldots op_{i-1}$. We prove the lemma for operation $op_i$ by case analysis.
If $op_i$'s location-based version $\langle v_i, j_i\rangle$ is maximal in $C$ (with respect to all linked cells and published entries with the same key) then \lp{op_i} is the last read retrieving $v_i$ from the global counter. This step is done after the put is published in the \code{ppa} (which is after $j_i$ is allocated) and before $\sigma_i$, and Conditions~\ref{proof:put:lp1}-\ref{proof:put:lp3} of the lemma hold. In addition, by the induction hypothesis, linearization points of $op_1, \ldots op_{i-1}$ preserve their location-based version order. They are all linearized in read steps of the global version counter returning their versions, specifically not later than \lp{op_i}---the latest read step returning the maximal version, hence Condition~\ref{proof:put:lp4} holds as well.

Otherwise, another operation $op_l$ published $\langle v_l, j_l\rangle$ in $\sigma_l$ before $\sigma_i$, s.t. $\langle v_l, j_l\rangle > \langle v_i, j_i\rangle$. By definition, $op_i$ is linearized exactly at the point (\lp{op_l}) which preserves the location-based version order of the operations, and Condition~\ref{proof:put:lp4} holds. By Condition~\ref{proof:put:lp3} of the induction hypothesis, \lp{op_l} is after an operation in $OP_k^c$ with location-based version equal or greater than $\langle v_l, j_l\rangle$ is published in $c$'s \code{ppa}. Since $\langle v_l, j_l\rangle > \langle v_i, j_i\rangle$, Condition~\ref{proof:put:lp3} also holds.

It is left to discuss Conditions~\ref{proof:put:lp1} and~\ref{proof:put:lp2}. Consider first the case where $v_l>v_i$. By Condition~\ref{proof:put:lp2} of the induction hypothesis \lp{op_l} is in a read step of the global version counter that occurred after it is set to $v_l$. $op_i$ eventually obtains the version $v_i$ which is smaller than $v_l$. This implies $op_i$ published the operation in the \code{ppa} before the version counter is set to $v_l$, and  \lp{op_i} satisfies Conditions~\ref{proof:put:lp1} and~\ref{proof:put:lp2}. If $v_l =v_i$ and $j_l>j_i$ then $op_l$ allocated $j_l$ after $op_i$ allocated $j_i$. By Condition~\ref{proof:put:lp1} of the induction hypothesis, \lp{op_l} is after $op_l$ allocated $j_l$  and before $\sigma_l$, and  \lp{op_i} satisfies Conditions~\ref{proof:put:lp1} and~\ref{proof:put:lp2}.
\end{proof}

%\par{Lemma~\ref{proof:get} Proof.}
\begin{proof}\textbf{of Lemma~\ref{proof:get}.}
It can be inferred from \code{findLatest} that $op$ returns the value with the maximal location-based version observed by $op$ in the \code{ppa} and in the cell linked list. In addition, it can be inferred from the code that put operations update values in-place in the cell linked list only if its location-based version is higher than the location-based version of the cell in the list.

First, assume $op$ did not find the key in $c$. By the observations above, all operations in $OP_k^c$ are published in the \code{ppa} after $\tau$. Otherwise, $op$ should have observed them either in the \code{ppa} or in the linked list. By Condition~\ref{proof:put:lp3} of Lemma~\ref{proof:put}, all operations in $OP_k^c$ are linearized after $\tau$, and Condition~\ref{proof:get:lp1} holds.

Next, assume $op$ returns the value written by operation $op_l$;
the location-based version of $op_l$ is $\langle v_l, j_l\rangle$. 

If $c$ is not accessible from the chunks list in the configuration preceding $\tau$, then \lp{op} is in the last step in which $c$ is accessible from the chunks list. By Claim~\ref{proof:rebalance:freezing} of Lemma~\ref{proof:rebalance} no put operation can publish its location-based version in $c$'s \code{ppa} after $\tau$, and by Condition~\ref{proof:put:lp1} of Lemma~\ref{proof:put}  \lp{op} is after  \lp{op_l}. Otherwise, it is clear by definition that \lp{op} is after  \lp{op_l}.

Consider an operation $op_m \in OP\setminus\{op_l\}$ with location-based version $\langle v_m, j_m\rangle$.
By Condition~\ref{proof:put:lp4} of Lemma~\ref{proof:put}, if $\langle v_m, j_m\rangle < \langle v_l, j_l\rangle$ then \lp{op_m} is before \lp{op_l}. It is left to show that if $\langle v_m, j_m\rangle > \langle v_l, j_l\rangle$ then \lp{op_m} is after $\tau$. By the observations above, all operations $op_m \in OP_k^c\setminus\{op_l\}$ with location-based version $\langle v_m, j_m\rangle > \langle v_l, j_l\rangle$ are published in the \code{ppa} after $\tau$, and hence are not observed by $op$. Condition~\ref{proof:put:lp3} of Lemma~\ref{proof:put} implies that these operations are linearized after at least one of them is published in the \code{ppa}, hence Condition~\ref{proof:get:lp2} also holds.
\end{proof}

%\par{Lemma~\ref{proof:scan} Proof.}
\begin{proof}\textbf{of Lemma~\ref{proof:scan}.}
It can be inferred from \code{findLatest} that for each key $k$ in the range of the scan, $op$ returns the maximal location-based version of $k$ that does not exceed the scan read point observed by $op$ in the \code{ppa} and in the cell linked list. In addition, it can be inferred from the code that put operations update values in-place in the cell linked list only if its location-based version is higher than the location-based version of the cell in the list.

Consider a put operation $op_m$ that writes to a key in the range of the scan but is not observed by $op$. If $op_m$ acquires version that is less than $v$ then it acquired a version before the scan increased the global version counter. Since $op$ did not observe $op_m$ in the \code{ppa}, $op_m$ completed before $op$ read the entry in the \code{ppa}, and since $op$ did not observe $op_m$ in the linked list, then the location-based version of the cell with key $k$ already in the linked list is higher than the location-based version of $op_m$.
By Condition~\ref{proof:put:lp4} of Lemma~\ref{proof:put}, $op_m$ is linearized before the put operation writing the value returned by the scan.

Finally, by Condition~\ref{proof:put:lp2} of Lemma~\ref{proof:put} all put operations that are not observed by $op$ and acquire version that is greater than $v$ are linearized after \lp{op}.
\end{proof}


\subsection{Progress}
\label{app:proof:progress}

\newcommand{\abs}[1]{\mid{#1}\mid}

We now show that \emph{get} and \emph{scan} operations are wait-free, and \emph{put} operations are lock-free.

We say that a sequence $\pi_s$ is a \emph{sub-execution}, if $\pi_s$ is a suffix (or a prefix) of an execution.
We say that thread $t$ is \emph{running} in a sub-execution $\pi_s$, if at least one step in $\pi_s$ is executed by thread $t$.
%
Given two keys $K_1$ and $K_2$, we write $\abs{K_2 - K_1}$ to denote the number of possible keys between $K_1$ and $K_2$ (notice that $\abs{K_2 - K_1}$ is always a finite number because each key is stored in a bounded number of bytes).
%
Given a chunk $X$, we write $min(X)$ to denote the minimal key in $X$ (as mentioned before, $min(X)$ is never changed during the lifetime of $X$).



\begin{lemma}
\label{proof:orderLemma}
Let $\pi = C_0,s_1,C_1, \ldots,s_i,C_i,\ldots$ be an execution.
Let $X$ and $Y$ be two chunks in configuration $C_i$ such that $Y=X.next$.
For any configuration $C_j$ such that $j \geq i$ we have  $min(X) < min(Y)$ in $C_j$.
\end{lemma}
\begin{proof}
The algorithm writes the address of $Y$ in $X.next$ only if $min(X) < min(Y)$
(this happens either at line $22$ of the function \emph{Kiwi::Replace}, or within the function \emph{balance()}).
Since $min(X)$ and $min(Y)$ are never changed, $min(X) < min(Y)$ in  any $C_j$ such that $j \geq i$.

\end{proof}

\begin{lemma}
\label{proof:aaaaaaaa}
The function \emph{get} is wait-free.
\end{lemma}
\begin{proof}
We prove that \emph{get} is wait-free by showing that the functions \emph{Kiwi::FindChunk} and \emph{Chunk::Find} are wait-free.

The function \emph{Kiwi::FindChunk} has a single loop: we write ${next}_i$ to denote the value of variable \emph{next} at the beginning of the $i$-th iteration of the loop.
%
Because of Lemma~\ref{proof:orderLemma}, $min({next}_i) < min({next}_{i+1})$.
Hence the loop terminates after at most $\abs{key - min({next}_1)}$ iterations.
%
Since all the functions invoked by \emph{Kiwi::FindChunk} are wait-free, each invocation of \emph{Kiwi::FindChunk} by thread $t$ completes within a finite number of steps by $t$.

The function \emph{Chunk::Find} invokes the functions \emph{Chunk::FindInPendingList} and \emph{Chunk::FindInCellList}.
The loop in \emph{Chunk::FindInPendingList} completes after a constant number of iterations.
The function \emph{Chunk::FindInCellList} goes over the chunk's link-list exactly once (this link-list has at most $M$ cells).
Hence, each invocation of \emph{Chunk::Find} by thread $t$ completes within a finite number of steps by $t$.

\end{proof}



\begin{lemma}
\label{proof:aaaaaaaa}
The function \emph{scan} is wait-free.
\end{lemma}

\begin{proof}
All functions invoked by \emph{scan} are wait-free.
It is sufficient to show that the loops in  \emph{scan} have
a finite number of iterations.

The function \emph{scan} has two nested loops:
we write $L_e$ to denote the external loop (begins at line $10$), and $L_i$ to denote the internal loop (begins at line $12$).
Let ${key}_i$ be the value of variable \emph{key} at the end of the $i$-th iteration of $L_e$.
Each ${key}_i$ is equal to the minimal key of the chunk which is handled by iteration $i+1$
(this chunk is pointed by the variable \emph{chunk}).
Because of Lemma~\ref{proof:orderLemma}, if $L_e$ has (at least) $i+1$ iterations then ${key}_i < {key}_{i+1}$.
%Since the chunks are sorted according to their minimal keys, ${key}_i < {key}_{i+1}$ for any $i$.
Hence, $L_e$ has at most  $\abs{maxKey - minKey}$ iterations.

The loop $L_i$ goes over the link-list of a chunk (this link-list has at most $M$ elements), hence $L_i$ has at most $M$ iterations.

Therefore, each invocation of \emph{scan} by thread $t$ completes within a finite number of steps by $t$.

\end{proof}





\begin{lemma}
\label{proof:aaaaaaaa}
The function \emph{put} is lock-free.
\end{lemma}
\begin{proof}
We prove by contradiction.
Assume that \emph{put} is not lock-free.
Hence there exists an infinite execution $\pi = C_0,s_1,C_1,\ldots$ such that
after configuration $C_{k0}$ ($k0 \geq 0$) no operation completes and no new operation is invoked.
%no operation completes after configuration $C_{k0}$.
%
We have already shown that \emph{get} and \emph{scan} are wait-free,
 therefore there exists configuration $C_{k1}$ in $\pi$ ($k0 \leq k1$) such that after $C_{k1}$ all the running threads execute  \emph{put} operations.


We write \emph{$t:X$.allocate} to denote an invocation of \emph{allocate} on chunk $X$ by thread $t$
(\emph{allocate} is invoked by function \emph{put} at line $6$).
We say that invocation \emph{t:$X$.allocate} is \emph{successful}
if it returns a reference to a valid cell (i.e., it returns a non-null value).

Consider an execution of the loop in function \emph{put} by thread $t$.
If \emph{$t:X$.allocate} is not successful at iteration $i+1$ ($i \geq 1$) of $t$ ,
then at iteration $i$ the invocation of \emph{$t:Y$.allocate} is also not successful.
Hence thread $t$ invokes \emph{Rebalance($Y$)} at the end of iteration $i$.
Therefore $Y \neq X$ and chunk $X$ has been added to kiwi (at some point) during iterations $i$ and $i+1$ of $t$
(when a chunk is added to kiwi, this chunk is not full).
Therefore another thread $t' \neq t$ successfully invoked \emph{$t':X$.allocate} (at some point) during iterations $i$ and $i+1$ of $t$
(otherwise \emph{$t:X$.allocate} should be successful).
%
Since a thread do not start new iteration of this loop after a successful invocation of \emph{allocate} (see lines $6$--$9$ in \emph{put}),
 we know that there exists configuration $C_{k2}$ in $\pi$ ($k1 \leq k2$) such that:
after $C_{k2}$ the \emph{put} operations do not start new iterations of this loop.

Since \emph{allocate} is wait-free, there exists configuration $C_{k3}$ in $\pi$ ($k2 \leq k3$) such that
no thread executes \emph{allocate} after $C_{k3}$.
Hence, no chunk becomes full in $\pi$ after configuration $C_{k3}$.


%Consider the loop in function \emph{put}.
%This loop may have iteration $i+1$ only if \emph{chunk.allocate} is not successful at iteration $i$ ---
%this can only happen if (at some point) after the beginning the $i$-th iteration a concurrent \emph{put} operation has successfully invoked \emph{chunk.allocate} on the same chunk.
%Hence there exists configuration $C_{k2}$ in $\pi$ ($k1 \leq k2$) such that: after $C_{k2}$ the \emph{put} operations do not start new iterations of this loop.

In the following paragraphs we focus on the functions \emph{Kiwi::Rebalance} and \emph{Kiwi::Replace}.
Notice that \emph{Kiwi::Rebalance} calls to \emph{Kiwi::Replace}; and that \emph{Kiwi::Replace} recursively calls to \emph{Kiwi::Rebalance} in line $20$.
An invocation of \emph{Kiwi::Rebalance} on chunk $X$ removes $X$ from kiwi:
hence for each thread $t$ and chunk $X$, \emph{Rebalance($X$)} may be invoked at most once by $t$.

Let $N_{puts}$ be the number of \emph{put} operations after configuration $C_{k3}$.
Let $N_E$ be the maximal number of new chunks created by an invocation of \emph{balance()} (in our implementation $N_E=4$).
Since after $C_{k3}$ no chunk may become full, at most $N_{puts} \times N_E$ new chunks may be created after $C_{k3}$.
Therefore, after configuration $C_{k3}$ the threads invoke function  \emph{Kiwi::Rebalance} a finite number of times.
Hence there exists a configuration $C_{k4}$ in $\pi$ ($k3 \leq k4$) such that \emph{Kiwi::Rebalance} is not invoked after $C_{k4}$.



%Let $N_{puts}$ be the number of \emph{put} operations after configuration $C_{k2}$.
%After $C_{k2}$, at most $N_{puts}$ chunks may become full.
%If $N_E$ is the maximal number of new chunks created by an invocation of \emph{balance()},
%then after configuration $C_{k2}$ at most $N_{puts} \times N_E$ new chunks may be created.
%Therefore, after configuration $C_{k2}$ the threads invoke function  \emph{Kiwi::Rebalance} a finite number of times.
%Hence there exists a configuration $C_{k3}$ in $\pi$ ($k2 \leq k3$) such that \emph{Kiwi::Rebalance} is not invoked after $C_{k3}$.

%Consider the loop in \emph{Kiwi::Replace}.

Since the other functions invoked by \emph{put} are lock-free\footnote{
The other functions invoked by \emph{put}, \emph{Kiwi::Rebalance} and \emph{Kiwi::Replace} are lock-free:
we assume that \emph{index.replace} is lock-free, the other ones are trivially wait-free.},
there exists a configuration $C_{k5}$ in $\pi$ ($k4 \leq k5$) such that all the running threads
after $C_{k5}$ execute the loop in \emph{Kiwi::Replace} (and this loop is never terminated).
%
Hence, the CAS at line $22$ always fails after configuration $C_{k5}$ (i.e., no CAS updates shared memory after $C_{k5}$).
This is a contradiction, because the  CAS at line $22$ may fail only if shared memory has been updated since the beginning of the iteration.



%
%
%
%Consider the loop in \emph{Kiwi::Replace}.
%Since the other functions are lock-free, there exists a configuration $C_{k4}$ in $\pi$ ($k3 \leq k4$) such that all the threads which are running after $C_{k4}$ do not exit from this loop.
%Hence, the CAS at line $22$ always fails after configuration $C_{k4}$ (i.e., no CAS updates shared memory after $C_{k4}$).
%This is a contradiction, because the  CAS at line $22$ may fail only if shared memory has been updated since the beginning of the iteration.

\end{proof}





\end{document}




