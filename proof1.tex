\newcommand{\lp}[1]{LP(\ensuremath{#1})}

\section{Correctness}
\label{sec:proof}

We now provide the key arguments for \kiwi's correctness. 
Due to space considerations, we state the main lemmas without proof, while 
a formal correctness proof is deferred to the full version of the paper.
We begin in Section~\ref{sec:spec} by defining the model and correctness notion we seek to prove, and then present  
the key safety arguments in Section~\ref{sec:safe}. We briefly touch on liveness in Section~\ref{sec:live}.

\subsection{Model and Correctness Specification}
\label{sec:spec}

We consider an asynchronous shared memory model~\cite{Welch2004}, where a finite number of threads execute memory \emph{operations} on shared objects. High-level objects, such as a map, are implemented using low-level memory objects supporting atomic read, write, and read-modify-write (e.g., CAS) operations. 
High-level operations are \emph{invoked}, then perform a sequence of  \emph{steps} on low-level objects, and finally \emph{return}.

Our correctness notion is \emph{linearizability}, which intuitively means that the object ``appears to be'' executing sequentially. 
It is defined for a \emph{history}, which is a sequence of operation invoke and return steps, possibly by multiple threads.
A history partially orders operations: 
operation $op1$ \emph{precedes} operation $op2$ in a history if $op1$'s return precedes $op2$'s invoke;
two operations that do not precede each other are \emph{concurrent}. In a  \emph{sequential} history, there are no concurrent operations.
An object is specified using a \emph{sequential specification}, which is the set of its allowed sequential histories. 
Roughly speaking, a history $\sigma$ is \emph{linearizable}~\cite{HerlihyW1990} if it has a sequential permutation that preserves 
$\sigma$'s precedence relation and satisfies the object's sequential specification. 
\remove{
For a history $\sigma$, \emph{complete($\sigma$)} is the sequence obtained by removing invocations with no responses from $\sigma$.
We assume that histories are \emph{well-formed}, meaning that the sub-sequence of each thread's steps in a history is sequential.
An object is \emph{linearizable}~\cite{HerlihyW1990} if each of its histories $\sigma$ can be extended by adding zero or more response events to a history $\sigma'$, so that  \emph{complete($\sigma'$)} has a sequential permutation that preserves $\sigma$'s precedence relation and 
satisfies the object's sequential specification. 
}

\kiwi\ implements a map offering put, get and scan operations, and in its sequential specification, get, and scan return the latest value inserted by a put for each key in their ranges.

\subsection{\kiwi's Linearizability}
\label{sec:safe}

Proving {\kiwi} is linearizable is accomplished by identifying, for every operation in a given history, a \emph{linearization point} between its invoke and return steps, so that the operation ``appears to'' occur atomically at this point. 
The linearization point of operation $op$ is denoted \lp{op}. 

\paragraph{Puts.} 

We saw above that puts in a chunk $C$
are ordered (lexicographically) according to their $\langle v, j \rangle$  pairs, where
$\langle v, i \rangle$ is published in their PPA in phase 2 of the put, and \code{$C$.k[$i$].valPtr$=j$}; this pair is called the \emph{full version} of the put.
We note that in each chunk, the full versions are unique, because threads obtain $j$ using F\&A.
First,  $i$ is published in \code{ppa[t].idx} (line~\ref{l:put-version}) and then
the location-version pair obtains its final value by a successful CAS of \code{ppa[t].ver}, either by the put (line~\ref{l:put-cas-version}) or by a helping thread (line~\ref {l:get-help}). 
We refer to 
a step publishing $i$ in \code{ppa[t].idx} and to
the step executing the successful CAS  as the put's \emph{publish time} and the \emph{full version assignment time}, resp., 
and say that the put \emph{assigned}  $\langle v, j \rangle$ for its key in $C$.

We note that each put assigns a full version at most once. 
Once a put operation $po$ for key $k$ assigns its full version in chunk $C$ at time $t$, we can define its linearization point. 
There are two options: 
\begin{enumerate}
\setlength{\itemsep}{0pt}
\setlength{\parskip}{0pt}
\item If at time $t$
$po$'s full version $\langle v, j \rangle$ is the highest for $k$ in $C$, (among entries in $C$'s {PPA} and  linked list),
then  \lp{po} is the last step reading $v$ from \code{GV} before $t$.
%%%% (line~\ref{l:put-LP}). 
\item  Otherwise, let $po'$ be the 
\code{put($k,\_$)} operation that assigns for $k$ in $C$ the smallest full version exceeding $po$'s
before time $t$. Then \lp{po} is recursively defined to be \lp{po'}. Note that 
$po$'s full version assignment time exceeds that of  $po'$, so the recursive definition does not induce cycles. 
\end{enumerate}
In case multiple puts are assigned to the same point, they are linearized in increasing full version order.

While a chunk is accessible from the chunks list its key range is well-defined.
We say that key $k$ is in the \emph{range} of chunk $C$ if $k \geq C$.\code{minKey} and $k < C.$\code{next.minKey}.
$C$ is \emph{mutable} if put operations can assign full version in $C$, otherwise it is \emph{immutable}.
An invariant of the rebalance process is that 
a chunk is immutable before it is accessible from the chunks list and after the freezing stage is completed. 
In addition, at any point in time each key is covered by the range of at most one mutable chunk. 
It is easy to show that a put operation always lands at a mutable chunk with a range that covers the key.
Thus, rebalance operations divide puts of key $k$ into disjoint groups; one group per mutable epoch of each chunk covering the key.
The following lemma establishes the order among linearization points of put operations within the same epoch.
\begin{lemma}
\label{proof:put}
Consider chunk $C$ accessible as of time $t_0$, key $k$ in the range of $C$, and an
operation $po=$\code{put($k, \_$)} that assigns $\langle v, j\rangle$ to $C$.\code{ppa} at time $t$. Then  
\begin{enumerate}
\setlength{\itemsep}{0pt}
\setlength{\parskip}{0pt}
\item \label{proof:put:lp1} \lp{po} is after $po$ allocates location $j$ for its value and before $t$.
\item \label{proof:put:lp2} \lp{po} is a read step of \code{GV} that returns $v$.
\item \label{proof:put:lp3} \lp{po} is after some operation $po'$ (possibly $po$, but not necessarily) publishes for $k$ to $C$ where later $po'$ assigns a full version equal to or greater than $\langle v, j\rangle$.
\item \label{proof:put:lp4} The linearization points of all operations that publish for $k$ to $C$ preserve their full version order.
\item \label{proof:put:lp5} At time $t_0$, the value published to $k$ by the put with the latest linearization point before $t_0$ is associated with the highest full version in C's linked list.
\end{enumerate}
\end{lemma}


%For every key k and time t when k is in the range of a chunk C that is accessible (either from the index or from the chunks list), the value published to k by the put with the latest LP before t is associated with the highest location-based version in C's linked list and PPA.


\paragraph{Gets and scans.}

The most subtle linearization is of get operations.
A get operation $go$ may land in a mutable or immutable chunk. 
We need to linearize $go$ before all concurrent puts that $go$ misses while seeking the value.
%get
For a get operation $go$ for a key $k$ in the range of chunk $C$, there are three options:
\begin{enumerate}
\setlength{\itemsep}{0pt}
\setlength{\parskip}{0pt}
\item If $C$ is not accessible from the chunks list when $go$ starts traversing $C$'s PPA, then \lp{go} is the last step in which $C$ is still accessible from the chunks list.
\item Else, if $go$ does not find $k$ in $C$ then \lp{go} is when $go$ starts traversing $C$'s PPA.
\item Else, let $po$ be the put operation that inserts the value returned by $go$. \lp{go} is the latest between when $go$ starts traversing $C$'s {PPA} and immediately after \lp{po}.
\end{enumerate}

The next lemma proves that in the third case no other put operation writing to $k$ is linearized after \lp{po} and before \lp{go}.
We rely on %Conditions~\ref{proof:put:lp1},~\ref{proof:put:lp3} and~\ref{proof:put:lp4} of 
Lemma~\ref{proof:put} and the rebalance invariants to prove it.

\begin{lemma}
\label{proof:get}
Consider a get operation $go$ retrieving the value of key $k$ from chunk $C$. Let $t$ be the step in which $go$ starts traversing $C$'s {PPA}. Then:
\begin{enumerate}
\setlength{\itemsep}{0pt}
\setlength{\parskip}{0pt}
\item \label{proof:get:lp1} If $go$ does not find $k$ in $C$, then for each operation $po$ publishing $k$ in $C$, \lp{po} is after $t$.
\item \label{proof:get:lp2} If $go$ returns the value written by operation $po$, then \lp{go} is after \lp{po}, and for each  $po' \neq po$ publishing $k$ in $C$, \lp{po'} is either before \lp{po} or after $t$.
\end{enumerate}
\end{lemma}

Scan operations are linearized when \code{GV} is increased beyond their read point, typically by their own F\&I, and sometimes by a helping rebalance. 
We rely on %Conditions~\ref{proof:put:lp2} and~\ref{proof:put:lp4} of 
Lemma~\ref{proof:put} to prove the following:
\begin{lemma}
\label{proof:scan}
Consider a scan operation $so$ that acquires version $v$ as its read point. For each key $k$ in the range of the scan, $so$ returns the value of the put operation writing to $k$ that is linearized last before \lp{so}.
\end{lemma}

The definition of the linearization points of scans and get operations imply that these operations are linearized between their invocation and return.
Condition~\ref{proof:put:lp1} of Lemma~\ref{proof:put} implies the same for puts. 
It is easy to show that gets and scans land in chunks that contain the saught keys in their ranges. Combined with the rebalancing invariants,
Lemma~\ref{proof:get} shows that get operations satisfy their sequential specification, and Lemma~\ref{proof:scan} proves that scans satisfy their sequential specification. 
Hence we conclude that \kiwi\ implements a linearziable map. 


\subsection{Liveness}
\label{sec:live}

 {\kiwi}'s gets and scans are \emph{wait-free}, namely, in any execution, every operation completes within a finite number of steps by its invoking thread. The proof simply shows that the number of iterations in the loops in these operations is finite. 

We further prove that put operations are \emph{lock-free}, namely, in every execution, \emph{some} operation completes. We show that although a put operation can execute an infinite number of rebalances, this occurs because 
some operation (and in fact many operations) successfully complete a put.

