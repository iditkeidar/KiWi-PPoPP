\section{Evaluation}
\label{sec:eval}

\subsection{Setup}

\begin{figure*}
\begin{center}
\begin{tikzpicture}
\begin{axis}[
        mystyle,
        unbalanced,
        title={(a) Get  },
        ylabel = { Throughput, M keys/sec}, 
        xlabel = { Threads\\ }
]
\addplot [black,mark=square*] table [x={threads}, y={KiWi}]
{results/get.txt}; 
\addplot [orange,mark=diamond] table [x={threads}, y={KAry}]
{results/get.txt}; 
\addplot [blues5,mark=10-pointed star] table [x={threads}, y={JavaSkipList}]
{results/get.txt}; 
\addplot [magenta,mark=+] table [x={threads}, y={SnapTree}]
{results/get.txt};
\end{axis}
\end{tikzpicture}
\begin{tikzpicture}
\begin{axis}[
        mystyle,
        unbalanced,
        title={ (b) Put  },
        xlabel = { Threads\\ }
]
\addplot [black,mark=square*] table [x={threads}, y={KiWi}]
{results/insert-delete.txt}; 
\addplot [orange,mark=diamond] table [x={threads}, y={KAry}]
{results/insert-delete.txt}; 
\addplot [blues5,mark=10-pointed star] table [x={threads}, y={JavaSkipList}]
{results/insert-delete.txt}; 
\addplot [magenta,mark=+] table [x={threads}, y={SnapTree}]
{results/insert-delete.txt};
\end{axis}
\end{tikzpicture}
%%%%%%%%%%%%%%%%%%%
%%%%%%%%%%%%%%%%
\begin{tikzpicture}
\begin{axis}[
	mystyle,
	scansOnly,
	title= {\small{ (c) Scan (32K keys) }},
        xlabel = { Threads\\ }
]
\addplot [black,mark=square*] table [x={threads}, y={KiWi}]
{results/scans32K.txt};
\addplot [orange,mark=diamond] table [x={threads}, y={KAry}]
{results/scans32K.txt};  
\addplot [blues5,mark=10-pointed star] table [x={threads}, y={JavaSkipList}]
{results/scans32K.txt}; 
\addplot [magenta,mark=+] table [x={threads}, y={SnapTree}]
{results/scans32K.txt};
\end{axis}
\end{tikzpicture}
\ref{scansOnlyLegend}
\end{center}
\caption{Throughput scalability with uniform workloads. (a) Get operations, (b) Put operations, and (c) Scan operations. }
\label{evaluation:results:getputscan}
\end{figure*}

\begin{figure*}
\begin{center}
\begin{tikzpicture}
\begin{axis}[
	mystyle16,
	unbalanced,
	title= {\small{(a) Scan 32K keys (parallel put), 1M}},
	ylabel = { Scan throughput, M keys/sec},
	xlabel= {Scan threads\\}
]
\addplot [black,mark=square*] table [x={threads}, y={KiWi}]
{results/put_scan32K.txt};
\addplot [orange,mark=diamond] table [x={threads}, y={KAry}]
{results/put_scan32K.txt};  
\addplot [blues5,mark=10-pointed star] table [x={threads}, y={JavaSkipList}]
{results/put_scan32K.txt}; 
\addplot [magenta,mark=+] table [x={threads}, y={SnapTree}]
{results/put_scan32K.txt};
\end{axis}
\end{tikzpicture}
%\hspace{.05\textwidth}
\begin{tikzpicture}
\begin{axis}[
	rangestyle_ext,
	title={\small{ (b) Scan 16 threads (parallel put), 1M}},
	xlabel= {Scan range (keys)\\}
]
\addplot [black,mark=square*] table [x={ranges}, y={KiWi}]
{results/scans_mixed_by_range.txt}; 
\addplot [orange,mark=diamond] table [x={ranges}, y={KAry}]
{results/scans_mixed_by_range.txt}; 
\addplot [blues5,mark=10-pointed star] table [x={ranges}, y={JavaSkipList}]
{results/scans_mixed_by_range.txt}; 
\addplot [magenta,mark=+] table [x={ranges}, y={SnapTree}]
{results/scans_mixed_by_range.txt};
\end{axis}
\end{tikzpicture}
\begin{tikzpicture}
\begin{axis}[
	rangestyle_ext,
	title={\small{ (c) Scan 16 threads (parallel put), 10M}},
	xlabel= {Scan range (keys)\\}
]
\addplot [black,mark=square*] table [x={ranges}, y={KiWi}]
{results/scans_bg_puts_by_range_10M.txt}; 
\addplot [orange,mark=diamond] table [x={ranges}, y={KAry}]
{results/scans_bg_puts_by_range_10M.txt}; 
\addplot [blues5,mark=10-pointed star] table [x={ranges}, y={JavaSkipList}]
{results/scans_bg_puts_by_range_10M.txt}; 
\addplot [magenta,mark=+] table [x={ranges}, y={SnapTree}]
{results/scans_bg_puts_by_range_10M.txt};
\end{axis}
\end{tikzpicture}

\begin{tikzpicture}
\begin{axis}[
	mystyle16,
	unbalanced,
	xlabel= { Put threads\\},
	ylabel = { Put throughput, M keys/sec },
	title={\small{(d) Put (parallel scan 32K keys), 1M}}
]
\addplot [black,mark=square*] table [x={threads}, y={KiWi}]
{results/s_put32K.txt};
\addplot [orange,mark=diamond] table [x={threads}, y={KAry}]
{results/s_put32K.txt};  
\addplot [blues5,mark=10-pointed star] table [x={threads}, y={JavaSkipList}]
{results/s_put32K.txt}; 
\addplot [magenta,mark=+] table [x={threads}, y={SnapTree}]
{results/s_put32K.txt};
\end{axis}
\end{tikzpicture}
%\hspace{.05\textwidth}
\begin{tikzpicture}
\begin{axis}[
	rangestyle_ext,
	xlabel= {Scan range (keys)\\},
	title= {\small{(e) Put 16 threads (parallel scan), 1M}}
]
\addplot [black,mark=square*] table [x={ranges}, y={KiWi}]
{results/puts_mixed_by_range.txt}; 
\addplot [orange,mark=diamond] table [x={ranges}, y={KAry}]
{results/puts_mixed_by_range.txt}; 
\addplot [blues5,mark=10-pointed star] table [x={ranges}, y={JavaSkipList}]
{results/puts_mixed_by_range.txt}; 
\addplot [magenta,mark=+] table [x={ranges}, y={SnapTree}]
{results/puts_mixed_by_range.txt};
\end{axis}
\end{tikzpicture}
\begin{tikzpicture}
\begin{axis}[
	rangestyle_ext,
	xlabel= {Scan range (keys)\\},
	title= {\small{(f) Put 16 threads (parallel scan), 10M}}
]
\addplot [black,mark=square*] table [x={ranges}, y={KiWi}]
{results/puts_bg_scans_by_range_10M.txt}; 
\addplot [orange,mark=diamond] table [x={ranges}, y={KAry}]
{results/puts_bg_scans_by_range_10M.txt}; 
\addplot [blues5,mark=10-pointed star] table [x={ranges}, y={JavaSkipList}]
{results/puts_bg_scans_by_range_10M.txt}; 
\addplot [magenta,mark=+] table [x={ranges}, y={SnapTree}]
{results/puts_bg_scans_by_range_10M.txt};
\end{axis}
\end{tikzpicture}

\ref{scansOnlyLegend}

\end{center}
\caption{Throughput scalability with concurrent scans and puts. (a,b) Scan operations. (c,d) Put operations. }
\label{evaluation:results:scan}
\end{figure*}

{\bf Implementation.} We implement {\kiwi} in Java, using Doug Lea's concurrent skip-list 
implementation~\cite{JavaConcurrentSkipList} for the index with added locks to support conditional updates. 
The code makes extensive use of efficient 
array copy methods~\cite{JavaArrayCopy}. {\kiwi}'s chunk size is set to 1024. 

The rebalance policy is tuned as follows:
$\code{checkRebalance}$  invokes rebalance with probability $0.15$ whenever
 the batched prefix consists of less than $0.625$ of the linked list. Rebalance 
engages the next chunk whenever engaging it will reduce the number of chunks in the list. 

We did not implement the piggybacking of puts on rebalance, and instead restart the put after every rebalance.
This does not violate lock-freedom because the number of threads is much smaller than the chunk size, and 
hence it is impossible for pending put operations to fill up an entire chunk.

\textbf{Methodology.}
We leverage the popular {\em synchrobench}  microbenchmark~\cite{Gramoli2015}
to exercise a variety of workloads. The hardware platform %is a high-performance server, 
features four Intel Xeon E5-4650 8-core CPUs. %(i.e., 32 hardware threads overall). %hyperthreading disabled). 
Every experiment starts with 20 seconds of {\em warmup} -- inserts
and deletes of random keys -- to let the HotSpot compiler optimizations take effect. 
It then runs 10  iterations, 5 seconds each, and averages the results. An iteration fills the map with 1M random (integer, integer) pairs, 
then exercises some workload. %scenario characterized by the selection of API's, their parameters, and degree of parallelism. 

\textbf{Competition.}
We compare {\kiwi} to Java implementations of three concurrent KV-maps: (1) the traditional 
skip-list~\cite{JavaConcurrentSkipList} which does not provide linearizable scan semantics, 
(2) {\kary}~\cite{BrownA12}\footnote{\small{\url{http://www.cs.toronto.edu/~tabrown/kstrq/LockFreeKSTRQ.java}}.}, 
and (3) {\snaptree}\cite{BronsonCCO2010}\footnote{\small{\url{https://github.com/nbronson/snaptree}}.}. 
%The implementations of {\kary} and {\snaptree} are available online. 
For {\kary}, we use the optimal configuration described in~\cite{BrownA12} with $k=64$. 

\subsection{Results}


\begin{figure}
\begin{center}
\begin{tikzpicture}
\begin{axis}[
	memstyle16,	
	title= {\small{Scan (32 keys) with background put }},
	xlabel = {Scan threads\\},
	ylabel = {Memory footprint (MB)}
]
\addplot [black,mark=square*] table [x={threads}, y={KiWi}]
{results/mem_p_s_r32.txt};
\addplot [orange,mark=diamond] table [x={threads}, y={KAry}]
{results/mem_p_s_r32.txt};  
\addplot [blues5,mark=10-pointed star] table [x={threads}, y={JavaSkipList}]
{results/mem_p_s_r32.txt}; 
\addplot [magenta,mark=+] table [x={threads}, y={SnapTree}]
{results/mem_p_s_r32.txt};
\end{axis}
\end{tikzpicture}

\ref{singleKeyOpsLegend}
\end{center}
\caption{RAM footprint with concurrent scans and puts. }
\label{evaluation:results:mem}
\end{figure}

\textbf{Basic scenarios: get, put, and scan.} 
We first focus on three simple workloads: 
(1) get-only (random reads), 
(2) put-only (random writes, half inserts/updates and half deletes), and 
(3) scan-only (sequential reads of 32K keys, each starting from a random lower bound).  
%The keys that define the operations are selected uniformly at random. 

Figure~\ref{evaluation:results:getputscan} depicts throughput scalability with the number of worker threads. 
In get-only scenarios (Figure~\ref{evaluation:results:getputscan}(a)), {\kiwi} outperforms the other 
algorithms by 1.25x to 2.5x. We explain this by the NUMA- and cache-friendly locality of access in its intra-chunk binary search. 
Under put-only workloads (Figure~\ref{evaluation:results:getputscan}(b)), it also performs well, thanks to avoiding version
manipulation. {\snaptree}, which is optimized for random writes, is approximately $10\%$ faster than {\kiwi}
with 32 threads. Note that in general, {\kiwi}'s gets are faster than its puts because the latter occasionally incur rebalancing. 

Finally, {\kiwi} excels in scan performance (Figure~\ref{evaluation:results:getputscan}(c)). 
For example, with 32 threads, it exceeds its closest competitor, {\kary}, by over $40\%$. 
Here too, {\kiwi}'s advantage stems from high locality of access while scanning big chunks. 

\textbf{Concurrent scans and puts.}
We now turn to the scenario that combines analytics (scan operations) with 
real-time updates (put operations). This is the primary use case that motivated 
the design principles behind {\kiwi}. Half of the threads perform scans, whereas 
the second half performs puts. 

Figure~\ref{evaluation:results:scan}(a) depicts scan throughput scalability with the number of threads
while scanning ranges of 32K keys. Figure~\ref{evaluation:results:scan}(b) depicts the throughput for 16 scan 
threads with varying range sizes. Note that for long scans, {\kary}'s performance deteriorates under contention. 
This happens because {\kary} restarts the scan every time a put conflicts with it -- i.e., puts make progress 
but scans get starved. For large ranges, {\snaptree} has the second-fastest scans because it shared-locks 
the scanned ranges in advance and iterates unobstructed. However, this comes at the expense of puts, 
since such locking starves concurrent updates. Figures~\ref{evaluation:results:scan}(c) 
and~\ref{evaluation:results:scan}(d) illustrate this phenomenon. 

We study the memory footprints of the solutions in this scenario. We focus on 32-key scans -- a setting 
in which the throughput achieved by all the algorithms except {\snaptree} is similar.  
Figure~\ref{evaluation:results:mem}(e) depicts the JVM memory-in-use metric immediately after a full 
garbage collection that cleans up all the unused objects, averaged across 50 data points. 
{\kiwi} is on par with {\kary} and the Java skiplist except with maximal parallelism (16 put threads), 
in which it consumes 20\% more RAM due to intensive version management.  

%Summing up, contrast to its competitors that are tailored for certain workloads, 
%{\kiwi} serves both scans and puts equally well, echoing the theoretical results.  

\textbf{Ordered workload.} 
As a balanced data structure, {\kiwi} provides good performance on non-random workloads. 
We experiment with a monotonically ordered stream of keys. {\kiwi} achieves a throughput 
similar to the previous experiments. In contrast, {\kary}'s maximal put throughput in this setting is 
730 times slower -- approximately $13.6$K operations/sec vs {\kiwi}'s $9.98$M. 
